\section{Definições e propriedades básicas}
\label{sec:propriedades}

 \todo{Nessa seção eu acrescentei as propriedades
 que o senhor havia mencionado e mudei a
 definição de ultrafiltro, prof.
 Além disso, para mostrar que toda família
 com a propriedade da interseção finita
 está em um ultrafiltro eu mostrei primeiro
 o Lema do Ultrafiltro.}

 %Podemos dar uma definição alternativa de vizinhanças
 %baseada na operação de interior.

%  \begin{definition}[Vizinhanças]
%  Dado um conjunto $X$ e $a \in X$, dizemos que 
%  $\mathcal{V}(a)$ 
%  é um conjunto de vizinhanças de $a$ se
 
%  \begin{enumerate}
%      \item $V \in \mathcal{V}(a) \implies a\in V$;
%      \item $V\in \mathcal{V}(a),
%      V \subset W \implies W \in \mathcal{V}(a)$;
%      \item $X \in \mathcal{V}(a)$;
%      \item $V, W \in \mathcal{V}(a) 
%      \implies V \cap W \in \mathcal{V}(a)$;
%      \item $V \in \mathcal{V}(a) 
%      \implies \interiorop (V) \in \mathcal{V}(a)$.
%  \end{enumerate}
 
%  \end{definition}

 Uma primeira observação interessante de se fazer
 é que a família de vizinhanças de 
 $a \in X$,
 $\mathcal{V}(a)$,
 é um filtro que tem
 $\tau (a)$
 como base, como mostraremos mais à frente.

 \begin{definition}[Filtro]
 Dado um conjunto $X$, dizemos que 
 $\mathfrak{F} \subset \powerset X$ 
 é um \textbf{filtro} em $X$ 
 se as seguintes propriedades são satisfeitas:

 \begin{enumerate}
    \item $\mathfrak{F} \neq \varnothing$;
    \item $\varnothing \notin \mathfrak{F}$;
    \item $V, W \in \mathfrak{F} 
    \implies V\cap W \in \mathfrak{F}$;
    \item $V \in \mathfrak{F}, V \subset W
    \implies W \in \mathfrak{F}$.
 \end{enumerate}

 \end{definition}
 
 \begin{proposition}
 O conjunto das vizinhanças de $a\in X$ é um filtro.
 \end{proposition}
 
 \begin{proof}
 Por definição, 
 $X \in \mathcal{V}(a)$,
 de modo que 
 $\mathcal{V}(a) \neq \varnothing$.
 Ademais, como
 $a\in V, \forall V \in \mathcal{V}(a)$,
 então nenhum elemento de 
 $\mathcal{V}(a)$
 é vazio, ou seja,
 $\varnothing \notin \mathcal{V}(a)$.
 A terceira e quarta propriedades dos filtros seguem
 diretamente da quarta e da segunda propriedade das
 vizinhanças, respectivamente.
 \end{proof}
 
 É importante notar que
 
 \begin{proposition}
 Dois conjuntos disjuntos não podem estar
 no mesmo filtro.
 \end{proposition}
 
 \begin{proof}
 Sejam 
 $A, B \subset X$
 disjuntos e
 $\mathfrak{F}$
 um filtro em 
 $X$.
 Suponha que
 $A, B \in \mathfrak{F}$.
 Ora, então por definição teríamos
 $A \cap B = \varnothing \in \mathfrak{F}$,
 absurdo.
 \end{proof}
 
 Consequentemente, como a definição a seguir mostrará,
 também não podemos ter conjuntos disjuntos no mesmo
 ultrafiltro.
 
 \begin{definition}[Ultrafiltro]
 Um \textbf{ultrafiltro} em
 $X$
 é um filtro
 $\mathfrak{U}$
 tal que os únicos filtros que o contêm são
 $\powerset X$
 e
 $\mathfrak{U}$.
 Dito de outro modo, um ultrafiltro é
 um \textbf{filtro maximal}.
 \end{definition}
 
 Usando a proposição anterior,
 uma maneira alternativa de enunciar a maximalidade,
 que nos dá uma outra definição clássica de ultrafiltro é
 ``um ultrafiltro em 
 $X$
 é um filtro
 $\mathfrak{U}$
 tal que para todo
 $Y \subseteq X$,
 ou
 $Y \in \mathfrak{U}$
 ou
 $X \setminus Y \in \mathfrak{U}$,
 não podendo acontecer ambos.''
 
 Ainda usando a proposição anterior, 
 temos a seguinte propriedade.
 
 \begin{proposition}
 Um ultrafiltro
 $\mathfrak{U}$
 em 
 $X$
 satisfaz a seguinte propriedade:
 dada uma partição qualquer
 \begin{equation*}
    X = X_1 \sqcup X_2
 \end{equation*}
 de $X$ em dois subconjuntos, 
 apenas um dos $X_i$ pertence a $\mathfrak{U}$. 
 Aqui,
 $\sqcup$
 denota união disjunta.
 \end{proposition}
 
 \begin{proof}
 Segue do fato de que conjuntos disjuntos não podem pertencer ao mesmo filtro.
 \end{proof}
 
 Interessantemente, não precisamos que
 $A$
 e
 $B$
 na proposição acima sejam disjuntos para
 concluirmos que um deles está no ultrafiltro
 (mas não concluiremos que \textbf{apenas um}
 deles está no filtro).
 De fato, melhor e mais útil é a proposição
 a seguir.
 
 \begin{proposition}
 Se 
 $\mathfrak{U}$ 
 é um ultrafiltro e 
 $X = A_1 \cup \dotsb \cup A_n$, 
 então existe
 $j$
 tal que 
 $A_j \in \mathfrak{U}$.
 \end{proposition}
 
 \begin{proof}
 Se nenhum dos
 $A_j$
 estivesse no ultrafiltro,
 então o complementar da união estaria em
 $\mathfrak{U}$.
 Ora, mas o complementar da união é
 $\varnothing$,
 absurdo.
 \end{proof}
 
 % 0. Toda família com a propriedade de 
 % interseção finita está contida em um ultrafiltro. 
 % (você usa sempre que diz "estenda isso para um
 % ultrafiltro que contém...")
 
 O lema a seguir é frequentemente chamado de
 \textbf{Lema do Ultrafiltro}, do inglês
 \textit{Ultrafilter Lemma}, e sua demonstração
 utiliza o axioma da escolha na forma do
 lema de Zorn.
 
 \begin{lemma}[Ultrafiltro]
 Todo filtro em
 $X$
 está contido em um ultrafiltro (em $X$).
 \end{lemma}
 % encontrei essa demonstração
 % em https://proofwiki.org/wiki/Ultrafilter_Lemma,
 % mas não estou muito confortável com ela...
 % a afirmação que a interseção está na união
 % eu que fiz, porque não vi sentido no que o
 % site fez nessa parte
 \begin{proof}
 Seja
 $\Omega$
 o conjunto dos filtros em
 $X$.
 A relação de ordem parcial
 $\subseteq$
 ordena
 $\Omega$
 parcialmente. 
 Tomando uma cadeia não vazia
 $\mathscr{C} \subseteq \Omega$,
 isto é, um subconjunto totalmente ordenado de
 $\Omega$,
 temos que
 $\displaystyle{ \bigcup \mathscr{C} } \in \Omega$
 e, portanto,
 $\displaystyle{ \bigcup \mathscr{C} }$
 é uma cota superior de
 $\mathscr{C}$.
 De fato, se
 $A,B \in \displaystyle{ \bigcup \mathscr{C} }$
 então existem filtros
 $\mathfrak{F}, \mathfrak{F}' \in \mathscr{C}$
 tais que
 $A \in \mathfrak{F}$
 e
 $B \in \mathfrak{F}'$.
 Como 
 $\mathscr{C}$
 é uma cadeia, podemos assumir s.p.g. que
 $\mathfrak{F} \subseteq \mathfrak{F}'$.
 Logo,
 $A \in \mathfrak{F}'$
 e, daí,
 $A \cap B \in \mathfrak{F}'$.
 Em particular,
 $A \cap B \in \displaystyle{ \bigcup \mathscr{C} }$.
 Pelo Lema de Zorn, 
 para todo
 $\mathfrak{F} \in \Omega$
 existe um elemento maximal 
 (com respeito à relação $\subseteq$)
 $\mathfrak{F}'$
 tal que
 $\mathfrak{F} \subseteq \mathfrak{F}'$.
 Ora, mas então por definição temos
 $\mathfrak{F}'$
 ultrafiltro, como desejado.
 \end{proof}
 
 \begin{corollary}
 Toda família com a propriedade de interseção finita
 está contida em um ultrafiltro.
 \end{corollary}

 \begin{proof}
 Seja 
 $X$
 um conjunto não vazio e
 $\mathcal{A} \subset \powerset X$
 uma família com a propriedade da interseção finita.
 Pelo Lema do Ultrafiltro,
 existe um ultrafiltro
 $\mathfrak{U}$
 em 
 $X$
 tal que
 $\mathcal{A} \subseteq \mathfrak{U}$.
 \end{proof}
 
 Sempre que dissermos
 ``estenda para um ultrafiltro que contém...'',
 esse corolário está sendo utilizado,
 ainda que implicitamente.
 Ademais, essa proposição e a propriedade de que
 conjuntos disjuntos não pertencem ao mesmo
 filtro tornam a caracterização de Hausdorff
 imediata!