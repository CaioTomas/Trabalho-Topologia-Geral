\section{Caracterização da topologia por ultrafiltros}
\label{sec:topol_filtros}
%Seção 2 da ref

 Seja 
 $X$
 um espaço topológico e 
 $\mathfrak{U}$
 um ultrafiltro em 
 $X$.
 Dizemos que 
 $\mathfrak{U}$
 \textbf{converge} para um ponto
 $x \in X$
 e denotamos
 $\mathfrak{U} \to x$
 se toda vizinhança aberta
 $U$
 de
 $x$
 é tal que
 $U \in \mathfrak{U}$.
 
 \begin{theorem}
 \label{teo:2.1}
 Seja
 $X$
 um espaço topológico.
 Então
 \begin{enumerate}[(a)]
    \item $X$ é Hausdorff se, e somente se, 
    todo ultrafiltro em $X$ converge para \textbf{no máximo} um ponto.
    \item $X$ é compacto se, e somente se, 
    todo ultrafiltro em $X$ converge para \textbf{pele menos} um ponto.
 \end{enumerate}
 \end{theorem}

 \begin{proof}
 \begin{enumerate}[(a)]
     \item Seja 
     $X$
     Hausdorff e suponha que
     $\mathfrak{U} \to x$
     e
     $\mathfrak{U} \to y$
     para dois pontos
     $x \neq y$.
     Sejam
     $U, V \subseteq X$
     vizinhanças abertas disjuntas de
     $x$
     e
     $y$,
     respectivamente. 
     Então temos
     $U, V \in \mathfrak{U}$
     e, portanto, 
     $\varnothing = U \cap V \in \mathfrak{U}$,
     absurdo.
     Logo,
     $\mathfrak{U}$
     converge para no máximo um ponto.
     
     Reciprocamente, suponha que todo ultrafiltro em
     $X$
     converge para no máximo um ponto, 
     e sejam
     $x \neq y$
     dois pontos de 
     $X$.
     Suponha que toda vizinhança aberta de 
     $x$
     tem interseção com toda vizinhança aberta de
     $y$.
     Então a família
     \begin{equation*}
         \mathfrak{G}
         =
         \setsuchthat{ \text{aberto } U \subseteq X }{ x \in U \text{ ou } y \in U }
     \end{equation*}
     tem a propriedade da interseção finita, de modo que
     $\mathfrak{G} \subseteq \mathfrak{U}$
     para algum ultrafiltro
     $\mathfrak{U}$.
     Se
     $U, V \subseteq X$
     são vizinhanças abertas de
     $x$
     e
     $y$, 
     respectivamente, então
     $U, V \in \mathfrak{U}$,
     de modo que
     $U \cap V \neq \varnothing$.
     
     \item Seja 
     $X$
     compacto e suponha que
     $\mathfrak{U}$
     é um ultrafiltro em
     $X$
     que não converge para nenhum ponto de
     $X$.
     Então para todo ponto
     $x \in X$,
     podemos encontrar uma vizinhança aberta
     $U_x$
     de
     $x$
     tal que
     $U_x \notin \mathfrak{U}$,
     ou seja,
     $X \setminus U_x \in \mathfrak{U}$.
     Obtemos então uma família de fechados
     \begin{equation*}
         \setsuchthat{ X \setminus U_x }{ x \in X }
     \end{equation*}
     com interseção vazia, já que
     $x \notin X \setminus U_x$.
     Daí, existem 
     $x_1, x_2, \dots, x_n \in X$
     tais que
     \begin{equation*}
         (X \setminus U_{x_1}) \cap \cdots \cap (X \setminus U_{x_2}) 
         = 
         \varnothing,
     \end{equation*}
     o que é impossível já que 
     $X \setminus U_{x_i} \in \mathfrak{U}$
     para todo 
     $i$.
     
     Reciprocamente, suponha que todo ultrafiltro em 
     $X$
     converge e suponha que
     $X$
     tem uma cobertura aberta
     $\mathcal{O}$
     que não possui subcobertura finita.
     Isso implica que a família de fechados
     \begin{equation*}
         \mathfrak{G}
         =
         \setsuchthat{ X \setminus U }{ U \in \mathcal{O} }
     \end{equation*}
     tem a propriedade da interseção finita.
     Estenda isso para um ultrafiltro
     $\mathfrak{U} \supseteq \mathfrak{G}$.
     Por hipótese, existe um ponto
     $x \in X$
     tal que 
     $\mathfrak{U} \to x$.
     O ponto 
     $x$
     é coberto por algum
     $U \in \mathcal{O}$,
     de modo que
     $U \in \mathfrak{U}$.
     Mas também temos
     $X \setminus U \in \mathfrak{G} \subseteq \mathfrak{U}$,
     de modo que
     $\mathfrak{U}$
     contém
     $(X \setminus U) \cap U = \varnothing$,
     uma contradição.
 \end{enumerate}
 \end{proof}
 
 Uma consequência do item (b) é que ``todo ultrafiltro em 
 $X$ 
 converge para pelo menos um ponto'' é equivalente a
 ``toda cobertura aberta de 
 $X$ 
 possui subcobertura finita''. 
 Portanto, definir compacidade como
 ``toda cobertura tem subcobertura finita''
 é equivalente a definir 
 ``todo ultrafiltro converge para pelo menos um ponto''.
 
 Ademais, note que o teorema acima nos diz que 
 $X$
 é Hausdorff e compacto se, e somente se, todo ultrafiltro em
 $X$
 converge para exatamente um ponto.
 Temos também a seguinte caracterização da topologia por ultrafiltros.

 %talvez tirar esse teorema
 \begin{theorem}
 \label{teo:2.2}
 Seja
 $X$
 um espaço topológico.
 $U \subseteq X$ 
 é aberto se, e somente se, 
 $U \in \mathfrak{U}$
 para todo ultrafiltro
 $\mathfrak{U}$
 convergindo para algum ponto de
 $U$.
 \end{theorem}
 \begin{proof}
 Se 
 $U \subseteq X$
 é aberto e
 $\mathfrak{U}$
 é um ultrafiltro convergindo para um ponto
 $u \in U$,
 então devemos ter
 $U \in \mathfrak{U}$
 pela definição da convergência
 $\mathfrak{U} \to u$.
 
 Reciprocamente, seja
 $U \subseteq X$
 e suponha que
 $U \in \mathfrak{U}$
 para todo ultrafiltro
 $\mathfrak{U}$
 convergindo para algum ponto de
 $U$.
 Se 
 $U$ 
 não é aberto, então existe
 $u \in U$
 tal que toda vizinhança aberta de 
 $u$
 contém pelo menos um ponto de
 $X \setminus U$.
 Isso implica que a coleção
 \begin{equation*}
     \mathfrak{G}
     =
     \left\{ X \setminus U \right\} \cup \left\{ \text{vizinhanças abertas de } u \right\}
 \end{equation*}
 tem a propriedade da interseção finita.
 Estenda essa família para um ultrafiltro
 $\mathfrak{U} \supseteq \mathfrak{G}$.
 Por construção, 
 $\mathfrak{U}$
 contém toda vizinhança aberta de 
 $u$,
 de modo que
 $\mathfrak{U} \to u$.
 Por hipótese, 
 $U \in \mathfrak{U}$,
 o que implica
 $\varnothing = (X \setminus U) \cap U \in \mathfrak{U}$,
 absurdo. 
 Logo, 
 $U$
 deve ser aberto, como afirmado.
 \end{proof}