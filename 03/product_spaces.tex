\section{Espaços produto com ultrafiltros}
\label{sec:esp_produto}
% Seção 4 ref
 %talvez tirar essa seção
 
 \todo{Eu lembro que o senhor disse que não
 precisava dessa seção, mas eu acho que seria
 legal se ela ficasse.
 Me ajudou a entender mais intuitivamente
 a topologia produto, e tem o Teorema 3.3.2
 que eu usei na demonstração do Tychonoff
 (tem um comentário sobre isso lá na demonstração).}
 
 Seja 
 $X_{\alpha}$
 uma família de espaços topológicos indexados por
 $\alpha \in A$.
 O \textbf{espaço produto}
 $X = \displaystyle{ \prod_{\alpha \in A} X_{\alpha} }$
 tem pontos da forma
 $x = (x_{\alpha})_{\alpha}$,
 sendo 
 $x_{\alpha} \in X_{\alpha}, \forall \alpha \in A.$
 Descreveremos a seguir a topologia natural neste conjunto,
 imposta pelas topologias individuais nos espaços
 $X_{\alpha}$.
 Antes, contudo, vamos deixar claro o que queremos dizer por um ponto
 $(x_{\alpha})_{\alpha}$.
 
 No caso de
 $|A| = n < \infty$,
 podemos tomar, sem perda de generalidade,
 $A = \{ 0, 1, 2, \dots, n-1 \}$.
 Nesse caso,
 \begin{equation*}
     X 
     = 
     X_0 \times X_1 \times \cdots \times X_{n-1}
     =
     \setsuchthat{ (x_0, x_1, \dots, x_{n-1}) }{ x_i \in X_i \forall i \in A }.
 \end{equation*}
 No caso particular em que
 $X_0 = X_1 = \cdots = X_{n-1}$,
 obtemos
 \begin{equation*}
     X 
     = 
     \underbrace{ X_0 \times X_0 \times \cdots \times X_0 }_{n \text{ vezes}}
     =
     X_0^n.
 \end{equation*}
 Todo ponto
 $x = (x_0, x_1, \dots, x_{n-1}) \in X$
 pode então ser visto como uma função
 $f: A \to X_0$.
 
 Prosseguindo para produtos enumeráveis,
 vamos usar
 $\omega = \{ 0, 1, 2, \dots \}$
 como nosso conjunto de índices.
 Agora, temos um produto enumerável de conjuntos dado por
 \begin{equation*}
     X
     =
     \prod_{i \in \omega} X_i
     =
     X_0 \times X_1 \times X_2 \times \cdots
     =
     \setsuchthat{ (x_0, x_1, x_2, \dots) }{ x_i \in X_i \forall i \in \omega }.
 \end{equation*}
 Aqui, cada ponto
 $x \in X$
 é uma sequência infinita,
 identificada com a correspondência
 $i \mapsto x_i$.
 Como casos especiais em que todos os espaços da família são iguais, 
 temos
 \begin{equation*}
     \reals^{\omega}
     =
     \reals \times \reals \times \reals \times \cdots,
 \end{equation*}
 o conjunto de todas as sequências infinitas enumeráveis de números reais,
 e
 \begin{equation*}
     2^{\omega}
     =
     \{ 0,1 \}^{\omega}
     =
     \{ 0,1 \} \times \{ 0,1 \} \times \{ 0,1 \} \times \cdots,
 \end{equation*}
 o conjunto de todas as sequências binárias infinitas enumeráveis.
 
 Para um conjunto $A$ de cardinalidade arbitrária,
 o produto geral
 $X = \displaystyle{ \prod_{\alpha \in A} X_{\alpha} }$
 pode ser pensado como o conjunto de todas as funções definidas em
 $A$
 que levam
 $\alpha \mapsto x_{\alpha} \in X_{\alpha}$
 para cada
 $\alpha \in A.$
 Podemos denotar os pontos por 
 $x = (x_{\alpha})_{\alpha \in A}$.
 Novamente considerando o caso particular em que todos os
 $X_{\alpha}$'s
 são iguais a um mesmo espaço
 $X_0$,
 temos
 \begin{equation*}
     X
     =
     X_0^A
     =
     \{ \text{funções } A \to X_0 \}.
 \end{equation*}
 Por exemplo, o conjunto
 $\reals^{\reals}$
 é simplesmente o conjunto de todas as funções
 $\reals \to \reals$.
 
 Todo espaço produto
 $X = \displaystyle{ \prod_{\alpha \in A} X_{\alpha} }$
 vem naturalmente equipado com projeções nos vários fatores;
 elas são as sobrejeções dadas por
 \begin{equation*}
     \functionarray{\pi_{\alpha}}{X}{X_{\alpha}}{x}{x_{\alpha}}.
 \end{equation*}
 Pensando em um ponto
 $x \in X$
 como uma função, a $\alpha$-ésima coordenada
 $\pi_{\alpha}(x) = x_{\alpha}$
 é simplesmente o valor daquela função no \textit{input}
 $\alpha \in A$.
 
 A partir daí, a topologia natural que mencionamos para o espaço produto
 é a chamada \textbf{topologia produto}, que é a topologia mais fraca
 na qual cada uma das projeções
 $\pi_{\alpha} : X \to X_{\alpha}$
 é contínua. 
 Equivalentemente, uma base para a topologia produto em 
 $X$
 é a coleção de abertos da forma
 \begin{equation*}
     \prod_{\alpha \in A} U_{\alpha}
 \end{equation*}
 onde cada
 $U_{\alpha} \subseteq X_{\alpha}$
 é aberto e
 $U_{\alpha} \neq X_{\alpha}$
 para um número finito de índices.
 
 Por outro lado, a \textbf{topologia caixa}
 \footnote{Tradução literal do termo em inglês \textit{box topology}.}
 no espaço produto 
 $X$
 é a topologia cuja base consiste de todos os conjuntos da forma
 \begin{equation*}
     \prod_{\alpha \in A} U_{\alpha}
 \end{equation*}
 onde
 $U_{\alpha} \subseteq X_{\alpha}$
 apenas.
 Se $|A| < \infty$, então as duas topologias coincidem.
 Entretanto, em geral a topologia caixa é mais forte que a topologia produto
 (em geral forte demais para ser útil).
 A menos que seja especificado, 
 tomaremos a topologia produto para o espaço produto $X$.
 
 Como um exemplo, considere a sequência de pontos
 $v_1, v_2, v_3, \dots \in [0,1]^{\omega}$
 dada por
 \begin{align*}
     v_1 &= (0, 1, 1, 1, 1, 1, \dots), \\
     v_2 &= (0, 0, 1, 1, 1, 1, \dots), \\
     v_3 &= (0, 0, 0, 1, 1, 1, \dots), \\
     v_4 &= (0, 0, 0, 0, 1, 1, \dots),
 \end{align*}
 e assim por diante. É natural supor que
 $v_n$
 converge para o ponto
 $\mathbf{0} = (0,0,0,0,0,0,\dots)$
 em
 $[0,1]^{\omega}$,
 e isso de fato é verdade na topologia produto:
 uma vizinhança aberta de 
 $\mathbf{0}$ 
 nessa topologia tem a forma
 \begin{equation*}
     U
     =
     U_0 \times U_1 \times U_2 \times \cdots \times U_{m-1} 
     \times [0,1] \times [0,1] \times [0,1] \times \cdots,
 \end{equation*}
 em que cada
 $U_i$
 é uma vizinhança aberta de
 $0$.
 Como
 $v_n \in U$
 sempre que 
 $n \geq m$,
 temos
 $v_n \to \mathbf{0}$.
 Na topologia caixa, por outro lado, a sequência
 $v_n$
 não converge: considere por exemplo a vizinhança
 \begin{equation*}
     U'
     =
     [0, 1/2] \times [0, 1/2] \times [0, 1/2] \times [0, 1/2] \times \cdots,
 \end{equation*}
 uma vizinhança aberta de
 $\mathbf{0}$
 na topologia caixa.
 Note que
 $v_n$
 nunca entra em
 $U'$,
 não importa quão grande seja
 $n$.
 Uma primeira justificativa para a não convergência seria que
 as coordenadas de
 $v_n$
 convergem ``pontualmente'', e não ``uniformemente'', para 
 $0$.
 Contudo, o problema na verdade é pior:
 considere a sequência
 \begin{align*}
     w_1 &= (0, 1, 1, 1, 1, 1, \dots), \\
     w_2 &= (0, 0, 1/2, 1/2, 1/2, 1/2, \dots), \\
     w_3 &= (0, 0, 0, 1/4, 1/4, 1/4, \dots), \\
     w_4 &= (0, 0, 0, 0, 1/8, 1/8, \dots),
 \end{align*}
 e assim por diante,
 que converge para
 $\mathbf{0}$
 na topologia produto.
 As coordenadas convergem ``uniformemente'' para 0,
 e o fazem razoavelmente rápido,
 mas
 $w_n$
 nunca entra em 
 \begin{equation*}
     [0,1] \times [0,1/2] \times [0,1/3] \times [0,1/5] \times [0,1/9] \times \cdots,
 \end{equation*}
 logo
 $w_n \not\to \mathbf{0}$
 na topologia caixa.
 Essa topologia contém abertos demais, i.e.,
 está muito próxima da topologia discreta para nos ser de grande utilidade.
 
 Generalizando os exemplos acima,
 vemos que uma sequência de pontos de
 $X = \displaystyle{ \prod_{\alpha \in A} X_{\alpha} }$
 converge se, e somente se, convergir ``coordenada a coordenada'':
 
 \begin{theorem}
 \label{teo:4.1}
 Seja 
 $x_n = (x_{n, \alpha})_{\alpha}$
 uma sequência de pontos em 
 $X = \displaystyle{ \prod_{\alpha} X_{\alpha}}$.
 Seja também 
 $a = (a_{\alpha})_{\alpha} \in X$.
 Então
 $x_n \to a$
 em $X$ se, e somente se,
 $x_{n, \alpha} \to a_{\alpha}$
 para todo $\alpha$ quando $n \to \infty$.
 \end{theorem}
 
 \begin{proof}
 Suponha primeiramente que 
 $x_n \to a$.
 Como cada projeção
 $\pi_{\alpha}$
 é contínua, isso implica que
 $x_{n, \alpha} = \pi(x_n) \to \pi_{\alpha}(a) = a_{\alpha}$
 para todo $\alpha$.
 
 Para mostrar a recíproca, basta considerarmos uma vizinhança aberta
 $U$
 de 
 $a \in X$.
 Esse conjunto tem a forma
 $\displaystyle{ \prod_{\alpha} U_{\alpha} }$,
 sendo cada
 $U_{\alpha}$
 uma vizinhança aberta de
 $a_{\alpha} \in X_{\alpha}$
 e
 $U_{\alpha} = X_{\alpha}$
 para todo 
 $\alpha \notin \{ \alpha_1, \alpha_2, \dots, \alpha_m \}$.
 Sob a hipótese de que
 $x_{\alpha, n} \to a_{\alpha}$
 para todo
 $\alpha \in A$,
 existem constantes
 $N_1, N_2, \dots, N_m$
 tais que para todo
 $i \in \{ 1, 2, \dots, m \}$
 temos
 $x_{n, \alpha_i} \in U_{\alpha_i}$
 sempre que
 $n > N_i$.
 Seja
 $N = \max\{ N_1, \dots, N_m \}$.
 Para todo
 $n > N$
 e para todo
 $\alpha \in A$,
 temos
 $x_{n, \alpha} \in U_{\alpha}$:
 se
 $\alpha \in \{ \alpha_1, \dots, \alpha_m \}$,
 esse fato segue da escolha de 
 $n > N \geq N_i$;
 do contrário, esse fato segue simplesmente porque
 $x_{\alpha} \in U_{\alpha} = X_{\alpha}$.
 \end{proof}
 
 A topologia produto também pode ser caracterizada 
 utilizando ultrafiltros, como mostra o teorema a seguir.
 
 \begin{theorem}
 \label{teo:4.2}
 Sejam
 $\mathfrak{U}$
 um ultrafiltro em
 $X = \displaystyle{ \prod_{\alpha} X_{\alpha}}$
 e
 $x = (x_{\alpha})_{\alpha} \in X$.
 Então
 $\mathfrak{U} \to x$
 se, e somente se
 $(\pi_{\alpha})_{*}(\mathfrak{U}) \to x_{\alpha}$
 em cada $X_{\alpha}$.
 \end{theorem}
 
 \begin{proof}
 Se
 $\mathfrak{U} \to x$,
 então pelo Teorema \ref{teo:3.2} para cada
 $\alpha \in A$
 temos
 $(\pi_{\alpha})_*(\mathfrak{U}) \to x_{\alpha}$.
 
 Reciprocamente, suponha que para cada
 $\alpha \in A$
 temos
 $(\pi_{\alpha})_*(\mathfrak{U}) \to x_{\alpha}$.
 Seja
 $u \subseteq X$
 uma vizinhança aberta de
 $x$.
 Devemos mostrar que 
 $U \in \mathfrak{U}.$
 Para isso, basta considerar abertos da sub base da forma
 $U = \pi_{\alpha}^{-1}(U_{\alpha})$
 onde cada
 $U_{\alpha} \subseteq X_{\alpha}$
 é uma vizinhança aberta de
 $x_{\alpha}$.
 Nesse caso,
 $U_{\alpha} \in (\pi_{\alpha})_*(\mathfrak{U})$,
 de modo que
 $U = \pi_{\alpha}^{-1}(U_{\alpha}) \in \mathfrak{U}$
 como desejado.
 Como $\mathfrak{U}$ é fechado por interseções finitas 
 e para a operação de contém, esse resultado vale para
 uma vizinhança aberta arbitrária de
 $x$,
 nos dando
 $\mathfrak{U} \to x$.
 \end{proof}