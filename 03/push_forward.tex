\section{Ultrafiltros \textit{push forward}}
\label{sec:push_forward}
%Prop 3.1, Teo 3.2 ref

 Nesta seção, introduzimos o \textit{push forward}
 de um ultrafiltro e provamos dois resultados sobre ele.
 \begin{proposition}
 \label{prop:imag_ultrafiltro}
 Sejam 
 $f:X \to Y$
 uma função qualquer entre conjuntos e
 $\mathfrak{U}$
 um ultrafiltro em $X$. Defina
 \begin{equation*}
     f_{*}(\mathfrak{U})
     =
     \setsuchthat{V \subseteq Y}{f^{-1}(V) \in \mathfrak{U}}.
 \end{equation*}
 Afirmamos que 
 $f_{*}(\mathfrak{U})$
 é um ultrafiltro em $Y$.
 \end{proposition}
 
 \begin{proof}
 Como
 $f^{-1}(\varnothing) = \varnothing \notin \mathfrak{U}$
 e
 $f^{-1}(Y) = X \in \mathfrak{U}$,
 temos que
 $\varnothing \notin f_*(\mathfrak{U})$
 e
 $Y \in f_*(\mathfrak{U})$.
 Se 
 $V_1, V_2 \in f_*(\mathfrak{U})$,
 então
 $f^{-1}(V_1 \cap V_2) = f^{-1}(V_1) \cap f^{-1}(V_2) \in \mathfrak{U}$,
 de modo que
 $V_1 \cap V_2 \in f_*(\mathfrak{U})$.
 Ademais, se
 $V_1 \in f_*(\mathfrak{U})$
 e
 $V_1 \subseteq V \subseteq Y$,
 então
 $f^{-1}(V) \supseteq f^{-1}(V_1) \in \mathfrak{U}$,
 de modo que
 $f^{-1}(V) \in \mathfrak{U}$
 e
 $V \in f_*(\mathfrak{U})$.
 Portanto, 
 $f_*(\mathfrak{U})$
 é um filtro em 
 $Y$. 
 Por fim, se
 $Y = Y_1 \sqcup Y_2$,
 então
 $X = f^{-1}(Y_1) \sqcup f^{-1}(Y_2)$,
 logo
 $f^{-1}(Y_i) \in \mathfrak{U}$
 para exatamente um dos
 $i$'s,
 nos dando
 $Y_i \in f_*(\mathfrak{U})$.
 Portanto,
 $f_*(\mathfrak{U})$
 é, de fato, um ultrafiltro em
 $Y$.
 \end{proof}
 
 O ultrafiltro
 $f_*(\mathfrak{U})$ 
 é o \textit{push forward} do ultrafiltro
 $\mathfrak{U}$,
 também chamado de \textbf{ultrafiltro imagem}.
 Essencialmente, essa proposição nos diz que
 imagem de ultrafiltro é também ultrafiltro 
 (e, na verdade, é possível mostrar ainda que
 imagem de filtro é também filtro).
 Note que se
 $f : X \to Y$
 e
 $g : Y \to Z$,
 então
 $(g \circ f)_*(\mathfrak{U}) = g_*(f_*(\mathfrak{U}))$
 é um ultrafiltro em 
 $Z$.
 A regra
 $(g \circ f)_* = g_* \circ f_*$
 é característica de mapas ``\textit{push forward}''.
 No caso de \textit{pullbacks}, a regra é
 $(g \circ f)^* = f^* \circ g^*$.
 
 \begin{theorem}
 \label{teo:3.2}
 Sejam 
 $(X, \tau_X), (Y, \tau_Y)$
 espaços topológicos e considere
 $f:X \to Y$
 qualquer. Afirmamos que
 $f$ é contínua se, e só se,
 $f_{*}(\mathfrak{U}) \to f(x)$
 em $Y$ para todo ultrafiltro
 $\mathfrak{U} \to x$ 
 em $X$.
 \end{theorem}
 
 \begin{proof}
 Suponha que 
 $f$
 é contínua e seja
 $\mathfrak{U}$
 um ultrafiltro em 
 $X$
 tal que
 $\mathfrak{U} \to x$.
 Para toda vizinhança aberta
 $V \subseteq Y$
 de
 $f(x)$,
 o conjunto
 $f^{-1}(V) \subseteq X$
 é uma vizinhança aberta de
 $x$.
 Temos
 $f^{-1}(V) \in \mathfrak{U}$
 e, portanto,
 $V \in f_*(\mathfrak{U})$,
 de modo que
 $f_*(\mathfrak{U}) \to f(x)$
 por definição.
 
 Reciprocamente, suponha que
 $f_*(\mathfrak{U}) \to f(x)$
 em 
 $Y$
 sempre que
 $\mathfrak{U}$
 é um ultrafiltro em
 $X$
 convergindo para
 $x$.
 Seja
 $V \subseteq Y$
 aberto.
 Para mostrar que a pré-imagem
 $f^{-1}(V) \subseteq X$
 é aberta, usaremos o Teorema \ref{teo:2.2}.
 Seja
 $\mathfrak{U}$
 um ultrafiltro qualquer em
 $X$
 convergindo para algum ponto
 $x \in f^{-1}(V)$.
 Por hipótese, 
 $f_*(\mathfrak{U}) \to f(x)$
 em
 $Y$.
 Como
 $V \subseteq Y$
 é uma vizinhança aberta de
 $f(x)$,
 temos
 $V \in f_*(\mathfrak{U})$,
 ou seja,
 $f^{-1}(V) \in \mathfrak{U}$.
 Pelo Teorema \ref{teo:2.2},
 $f^{-1}(V)$
 é aberto e, portanto,
 $f$
 é contínua.
 \end{proof}
 
 O Teorema \ref{teo:3.2} suscita uma comparação entre
 ultrafiltros e sequências.
 Recorde que para uma função contínua
 $f : X \to Y$,
 temos
 $f(x_n) \to f(x)$
 em
 $Y$
 sempre que
 $x_n \to x$
 em 
 $X$.
 A recíproca não vale sem que adicionemos algumas hipóteses 
 (por exemplo, que $X$ seja espaço métrico).
 Sequências da forma
 $x_1, x_2, x_3, \dots$
 são limitadas no sentido de que elas são, por definição, enumeráveis.
 Ultrafiltros, que não possuem tal restrição, são suficientes para 
 caracterizar topologias e continuidade em geral.
 
 %falar de filtros gerados
 \begin{theorem}[Filtros gerados]
 
 \todo{Eu pesquisei e coloquei esse teorema
 sobre filtros gerados, prof, mas não pensei
 ainda como ele poderia simplificar as coisas.}
 
 Sejam
 $X$
 um conjunto e
 $\mathcal{B} \subset \powerset X$.
 Temos que
 \begin{equation*}
     \mathfrak{F}
     =
     \setsuchthat{V \subseteq X}{\exists U \in \mathcal{B}, U \subseteq V}
 \end{equation*}
 é um filtro em 
 $X$
 se, e somente se,
 \begin{enumerate}
     \item $\forall V_1, V_2 \in \mathcal{B}, \exists U \in \mathcal{B} : U \subseteq V_1 \cap V_2$;
     \item $\varnothing \notin \mathcal{B}$;
     \item $\mathcal{B} \neq \varnothing$.
 \end{enumerate}
 Nesse caso, 
 $\mathcal{B}$
 é uma \textbf{base para o filtro}
 $\mathfrak{F}$,
 que é dito \textbf{gerado} por
 $\mathcal{B}$.
 \end{theorem}
 
 % demonstração de
 % https://proofwiki.org/wiki/Filter_Basis_Generates_Filter
 \begin{proof}
 Assumamos primeiro que
 $\mathfrak{F}$
 é um filtro em
 $X$.
 Temos então
 $X \in \mathfrak{F}$
 e, portanto,
 $\mathcal{B} \neq \varnothing$.
 Ademais, como
 $\varnothing \notin \mathfrak{F}$,
 temos que
 $\varnothing \notin \mathcal{B}$
 já que
 $\mathcal{B} \subseteq \mathfrak{F}$.
 Agora, sejam
 $V_1, V_2 \in \mathcal{B}$.
 Então 
 $V_1, V_2 \in \mathfrak{F}$
 e, portanto,
 $V = V_1 \cap V_2 \in \mathfrak{F}$
 já que 
 $\mathfrak{F}$
 é um filtro.
 A definição de
 $\mathfrak{F}$
 implica, então, que existe
 $U \subseteq \mathcal{B}$
 tal que 
 $U \subseteq V = V_1 \cap V_2$.
 
 Agora, assumamos que 
 $\mathcal{B}$
 satisfaz as três condições do enunciado.
 Para mostrar que 
 $\mathfrak{F}$
 é um filtro, note que como
 $\mathcal{B} \neq \varnothing$
 então existe 
 $B \in \mathcal{B}$.
 Como 
 $\mathcal{B} \subseteq \powerset X$,
 então
 $B \subseteq X$
 e daí, por definição de filtro,
 $X \in \mathfrak{F}$.
 Ademais, como
 $\varnothing \notin \mathcal{B}$,
 segue que
 $\varnothing \notin \mathfrak{F}$.
 Por fim, sejam
 $V_1, V_2 \in \mathfrak{F}$.
 Existem 
 $U_1, U_2 \in \mathcal{B}$
 tais que
 $U_1 \subseteq V_1$
 e
 $U_2 \subseteq V_2$.
 Além disso, existe
 $U \in \mathcal{B}$
 tal que
 $U \subseteq U_1 \cap U_2$,
 donde segue que
 $U \subseteq V_1 \cap V_2$
 e, portanto,
 $V_1 \cap V_2 \in \mathfrak{F}$.
 \end{proof}