\section{O teorema da sub-base de Alexander}
\label{sec:alexander}

 \todo{Nessa seção eu ainda estou com as demonstrações
 inalteradas, tem de justificar melhor o uso do
 Lema de Zorn na demonstração do Teorema da Sub-Base.
 Além disso, também vou pensar em como fazer
 com filtros e sem o Lema de Zorn.}
 
 \begin{theorem}[Alexander]
 \label{teo:alexander}
 Seja
 $(X, \tau)$
 um espaço topológico.
 Se 
 $X$
 tem uma sub base
 $\mathcal{C}$
 tal que toda cobertura de
 $X$
 por elementos de
 $\mathcal{C}$
 tem uma subcobertura finita, então
 $X$
 é compacto.
 \end{theorem}
 
 % pensar em como fazer com filtros e sem
 % lema de Zorn
 
 % enxugar a demonstração com a do Matheus
 % lá do fórum
 
 % prova original enxugada
 \begin{proof}
 Suponha que 
 $X$
 não é compacto.
 Seja
 $\mathscr{F}$
 a coleção de todas as coberturas abertas de
 $X$
 que não têm subcobertura finita.
 Note que
 $\mathscr{F} \neq \varnothing$
 pois
 $X$
 não é compacto,
 e que essa família é parcialmente ordenada
 pela relação de inclusão.
 Daí, o Lema de Zorn garante que existe
 $\mathcal{M} \subseteq \mathscr{F}$
 maximal (com respeito à relação de inclusão).
 
 De fato, seja
 \begin{equation*}
     \mathscr{F}'
     =
     \setsuchthat{
     \mathcal{F}_{\lambda}
     }
     {
     \lambda \in \Lambda
     } \subseteq \mathscr{F}
 \end{equation*}
 um conjunto totalmente ordenado. Defina
 \begin{equation*}
     \mathcal{F} 
     = 
     \bigcup_{\Lambda} \mathcal{F}_{\lambda}.
 \end{equation*}
 Temos que 
 $\mathcal{F}$
 é uma cota superior de 
 $\mathscr{F}'$
 pois contém todo
 $\mathcal{F}_{\lambda}$
 e também que
 $\mathcal{F}$
 é uma cobertura aberta de
 $X$.
 
 Se tivéssemos
 $\mathcal{F} \notin \mathscr{F}'$,
 então existiriam 
 $F_1, \dots, F_n \in \mathcal{F}$
 cobrindo 
 $X$.
 Ora, mas cada um dos
 $F_j$ 
 pertence a algum dos
 $\mathcal{F}_{\lambda}$.
 Digamos que
 $F_j \in \mathcal{F}_{\lambda_j}$.
 Como
 $\mathscr{F}'$
 é totalmente ordenado,
 então
 $\{ \mathcal{F}_{\lambda_1}, \dots, \mathcal{F}_{\lambda_n} \}$
 também é.
 Sem perda de generalidade, suponhamos que
 $\mathcal{F}_{\lambda_n}$
 contém todos os
 $F_j$.
 Ora, então
 $\mathcal{F}_{\lambda_n}$
 é um elemento de
 $\mathscr{F}'$
 que contém uma subcobertura finita, ou seja,
 $\mathscr{F}'$
 tem uma subcobertura finita de 
 $X$, 
 absurdo.
 Portanto, 
 $\mathcal{F} \in \mathscr{F}'$.
 
 Seja 
 $\mathcal{S} = \mathcal{M} \cap \mathcal{C}$
 o conjunto dos abertos de
 $\mathcal{M}$
 que estão na sub-base.
 Vamos mostrar que
 $\mathcal{S}$
 cobre
 $X$,
 o que implicará que
 $\mathcal{M}$
 deve conter uma subcobertura finita de
 $X$,
 já que
 $\mathcal{S} \subset \mathcal{C}$
 e
 $\mathcal{S} \subseteq \mathcal{M}$,
 e, daí, teremos a conclusão absurda de que
 $\mathcal{M} \notin \mathscr{F}$,
 concluindo então que 
 $X$
 é, de fato, compacto.
 
 Suponhamos então que
 $\mathcal{S}$
 não cobre 
 $X$.
 Existe, então,
 $x \in X \setminus \bigcup \mathcal{S}$.
 Como
 $\mathcal{M}$
 cobre
 $X$, 
 existe
 $U \in \mathcal{M}$
 tal que
 $x\in U$
 e, como
 $\mathcal{C}$
 é sub-base, existem
 $C_1, \dots, C_n \in \mathcal{C}$
 tais que
 $C = \displaystyle{\bigcap_{i}^n C_i } \subseteq U$.
 Note que nenhum dos
 $C_i$
 está em
 $\mathcal{M}$,
 pois se um deles estivesse, teríamos
 $x\in C \in \mathcal{M}$,
 contrariando o fato de que 
 $x$
 não é coberto por 
 $\mathcal{S}$.
 
 Como 
 $\mathcal{M}$
 é maximal, segue que cada uma das coleções
 $\mathcal{M} \cup \{ C_i \}$
 possui subcobertura finita de
 $X$.
 Denote-a por
 $\mathcal{M}_i \cup \{ C_i \}$,
 sendo
 $\mathcal{M}_i \subseteq \mathcal{M}$.
 Temos, então, que
 \begin{equation*}
     \bigcup_i \mathcal{M}_i \cup \{ C \}
 \end{equation*}
 é uma cobertura aberta finita de 
 $X$,
 de modo que
 \begin{equation*}
     \bigcup_i \mathcal{M}_i \cup \{ U \}
 \end{equation*}
 também o é, já que
 $C \subseteq U$.
 Mas
 $U \in \mathcal{M}$,
 absurdo.
 
 Portanto,
 $\mathcal{S}$
 deve cobrir 
 $X$
 e, consequentemente,
 $X$
 é compacto.
 \end{proof}
 
 % prova original
%  \begin{proof}
%  O argumento é por absurdo.
%  Suponhamos que
%  $X$
%  não é compacto
%  (de modo que $X$ é infinito),
%  mas que toda cobertura sub básica por elementos de
%  $\mathcal{S}$
%  tem uma subcobertura finita.
%  Denote por
%  $\mathbb{S}$
%  o conjunto de todas as coberturas abertas de 
%  $X$
%  que não têm subcobertura finita de
%  $X$.
%  Ordene 
%  $\mathbb{S}$
%  parcialmente usando a relação de inclusão de conjuntos e
%  aplique o Lema de Zorn para encontrar um elemento
%  $\mathscr{C} \in \mathbb{S}$
%  maximal. 
%  Note que
%  \begin{enumerate}
%      \item como
%      $\mathscr{C} \in \mathbb{S}$,
%      então por definição de
%      $\mathbb{S}$
%      temos que
%      $\mathscr{C}$
%      é uma cobertura aberta de
%      $X$
%      que não tem subcobertura finita de 
%      $X$
%      (isso implica, em particular, que $\mathscr{C}$ é infinito)
     
%      \item a maximalidade de
%      $\mathscr{C}$
%      em
%      $\mathbb{S}$
%      implica que se
%      $V$
%      é um aberto de
%      $X$
%      tal que
%      $V \notin \mathscr{C}$,
%      então
%      $\mathscr{C} \cup \{ V \}$
%      tem subcobertura finita, que deve necessariamente ser da forma
%      $\{ V \} \cup \mathscr{C}_V$
%      para algum subconjunto finito
%      $\mathscr{C}_V$
%      de
%      $\mathscr{C}$.
%      Como sugere a notação, esse subconjunto finito depende de
%      $V$.
%  \end{enumerate}
%  Começaremos mostrando que
%  $\mathscr{C} \cap \mathcal{S}$
%  \textbf{não} cobre
%  $X$.
%  Suponhamos o contrário, o que implica, em particular, que
%  $\mathscr{C} \cap \mathcal{S}$
%  é uma cobertura de 
%  $X$
%  por elementos de
%  $\mathcal{S}$.
%  A hipótese do teorema em
%  $\mathcal{S}$
%  implica que existem um subconjunto finito de
%  $\mathscr{C} \cap \mathcal{S}$
%  que cobre
%  $X$,
%  que também é, simultaneamente, uma subcobertura finita de
%  $X$
%  por elementos de
%  $\mathscr{C}$,
%  já que
%  $\mathscr{C} \cap \mathcal{S} \subseteq \mathscr{C}$.
%  Mas isso contradiz o fato de que
%  $\mathscr{C} \in \mathcal{S}$,
%  provando que
%  $\mathscr{C} \cap \mathcal{S}$
%  não cobre 
%  $X$.
 
%  Como 
%  $\mathscr{C} \cap \mathcal{S}$
%  não cobre
%  $X$,
%  existe
%  $x \in X$
%  que não é coberto por
%  $\mathscr{C} \cap \mathcal{S}$,
%  ou seja, tal que
%  $x$
%  não pertence a nenhum elemento de
%  $\mathscr{C} \cap \mathcal{S}$.
%  Mas como
%  $\mathscr{C}$
%  cobre
%  $X$,
%  também existe
%  $U \in \mathscr{C}$
%  tal que
%  $x \in U$.
%  Como
%  $\mathcal{S}$
%  é uma sub base geradora da topologia de
%  $X$,
%  segue da definição de topologia gerada que
%  existe uma coleção finita de abertos da sub base
%  $S_1, \dots, S_n \in \mathcal{S}$
%  tal que
%  \begin{equation*}
%      x \in S_1 \cap \cdots \cap S_n \subseteq U.
%  \end{equation*}
%  Agora, vamos mostrar por absurdo que
%  $S_i \notin \mathscr{C}$
%  para todo
%  $i = 1, 2, \dots, n$.
%  Se 
%  $i$
%  fosse tal que 
%  $S_i \in \mathscr{C}$,
%  então teríamos
%  $S_i \in \mathscr{C} \cap \mathcal{S}$,
%  de modo que
%  $x \in S_i$
%  implicaria que
%  $x$
%  é coberto por
%  $\mathscr{C} \cap \mathcal{S}$,
%  contradizendo a maneira como 
%  $x$
%  foi escolhido.
 
%  Como mencionado anteriormente, a maximalidade de
%  $\mathscr{C}$
%  em
%  $\mathcal{S}$
%  implica que para todo
%  $i = 1, 2, \dots, n$, 
%  existe um subconjunto finito
%  $\mathscr{C}_{S_i}$
%  de
%  $\mathscr{C}$
%  tal que
%  $\{ S_i \} \cup \mathscr{C}_{S_i}$
%  é uma cobertura de 
%  $X$.
%  Defina
%  \begin{equation*}
%      \mathscr{C}_F 
%      = 
%      \mathscr{C}_{S_1} \cup \cdots \cup \mathscr{C}_{S_n},
%  \end{equation*}
%  que é um subconjunto finito de
%  $\mathscr{C}$
%  pois cada um dos
%  $\mathscr{C}_{S_i}$
%  o são. 
%  Note que para cada
%  $i = 1, 2, \dots, n$,
%  a união
%  $\{ S_i \} \cup \mathscr{C}_F$
%  é uma cobertura finita de 
%  $X$. 
%  Portanto, trocaremos os
%  $\mathscr{C}_{S_i}$
%  por
%  $\mathscr{C}_F$.
 
%  Denote por
%  $\bigcup \mathscr{C}_F$
%  a união de todos os conjuntos em
%  $\mathscr{C}_F$
%  (que é um aberto de $X$)
%  e denote
%  $Z = X \setminus \bigcup \mathscr{C}_F$.
%  Observe que para qualquer
%  $A \subseteq X$, 
%  temos que
%  $\{ A \} \cup \bigcup \mathscr{C}_F$
%  cobre
%  $X$
%  se, e somente se,
%  $Z \subseteq A$.
%  Em particular, para todo
%  $i = 1, 2, \dots, n$,
%  o fato de que
%  $\{ S_i \} \cup \bigcup \mathscr{C}_F$
%  cobre
%  $X$
%  implica que
%  $Z \subseteq S_i$.
%  Portanto,
%  $Z \subseteq S_1 \cap \cdots \cap S_n$.
%  Lembrando que
%  $S_1 \cap \cdots \cap S_n \subseteq U$,
%  temos
%  $Z \subseteq U$,
%  o que é equivalente a
%  $\{ U \} \cup \bigcup \mathscr{C}_F$
%  ser uma cobertura de
%  $X$.
%  Ademais, 
%  $\{ U \} \cup \bigcup \mathscr{C}_F$
%  é uma cobertura finita de 
%  $X$
%  tal que
%  $\{ U \} \cup \bigcup \mathscr{C}_F \subseteq \mathscr{C}$.
%  Portanto, 
%  $\mathscr{C}$
%  tem uma subcobertura finita de
%  $X$,
%  contradizendo o fato de que
%  $\mathscr{C} \in \mathbb{S}.$
%  Logo, a suposição inicial de que
%  $X$
%  não era compacto está incorreta, mostrando que
%  $X$
%  é compacto.
%  \end{proof}
 
 Note que a recíproca do teorema também vale:
 se
 $X$
 é compacto e 
 $\mathcal{S}$
 é uma sub base de
 $X$,
 então toda cobertura de
 $X$
 por elementos de
 $\mathcal{S}$
 tem subcobertura finita.
 Ela pode ser demonstrada tomando
 $\mathcal{S} = \tau$
 (já que toda topologia é sub base de si mesma).
 
 Com esse teorema,
 podemos demonstrar o Teorema \ref{teo:alexander}
 de maneira alternativa, sem filtros, como segue.
 
 \begin{proof}[Prova alternativa do Teorema \ref{teo:tychonoff}]
 A topologia produto em
 $X$
 tem, por definição, uma sub base que consiste de
 conjuntos ``cilindros'', que são as imagens inversas
 pelas projeções de abertos em cada fator.
 Dada uma família
 $C$
 de elementos da sub base de
 $X$
 que não tem subcobertura finita,
 podemos particionar
 $C = \displaystyle{ \bigcup_{\alpha} C_{\alpha} }$
 em subfamílias que consistem exatamente dos conjuntos
 cilindros correspondentes a um dado fator de
 $X$.
 Por hipótese, se
 $C_{\alpha} \neq \varnothing$,
 então
 $C_{\alpha}$
 não tem subcobertura finita.
 Por serem conjuntos cilindros, isso significa que
 suas projeções em
 $X_{\alpha}$
 não tem subcobertura finita, e como cada
 $X_{\alpha}$
 é compacto, podemos encontrar
 $x_{\alpha} \in X_{\alpha}$
 que não é coberto pelas projeções de
 $C_{\alpha}$
 em
 $X_{\alpha}$.
 Mas então
 $(x_{\alpha})_{\alpha} \in \displaystyle{ \prod_{\alpha} X_{\alpha} }$
 não é coberto por
 $C$, 
 absurdo.
 \end{proof}