\section{Compacidade com filtros}
\label{sec:compac_filtros}

 %mostrar equivalência entre as diferentes definições
 
 A definição ``standard'' é dada 
 em termos de coberturas, e foi usada
 na demonstração do Teorema \ref{teo:2.1}.
 De fato, o item (b) desse teorema nos mostra que
 uma definição alternativa de compacidade em termos
 de filtros poderia ser
 
 \begin{definition}[Compacidade com filtros]
 \label{def:compac_filtros}
 Todo ultrafiltro converge para pelo menos
 um ponto.
 \end{definition}
 
 %Se F é um ultrafiltro, então existe x tal que V(x) \subset F.
 
%  \begin{definition}[Compacidade com abertos]
%  \label{def:compac_abertos}
%  Dizemos que o espaço topológico
%  $X$
%  é compacto (com respeito a uma determinada topologia)
%  se toda cobertura aberta
%  $\mathcal{A}$
%  de 
%  $X$
%  possui subcobertura finita.
%  \end{definition}