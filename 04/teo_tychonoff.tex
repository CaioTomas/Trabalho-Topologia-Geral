\section{O teorema de Tychonoff com filtros}
\label{sec:tychonoff}

 \begin{theorem}[Thychonoff]
 \label{teo:tychonoff}
 Dada uma família
 $X_{\alpha}$
 de espaços topológicos compactos,
 o espaço produto
 $X = \displaystyle{ \prod_{\alpha} X_{\alpha} }$
 também é compacto com respeito à topologia produto.
 \end{theorem}
 
 Apresentamos aqui uma demonstração utilizando filtros.
 Em seguida, apresentamos e demonstramos o Teorema
 da sub-base de Alexander e utilizamos esse resultado
 para dar um prova alternativa do Teorema de Tychonoff.
 
 \begin{proof}
 \todo{Aqui eu coloquei uma demonstração
 que aproveita a caracterização de compacidade
 com filtros que eu tinha colocado antes.
 Não sei se era exatamente isso que você
 tinha em mente, prof, mas mesmo que não
 seja achei o argumento muito elegante :-)}
 % demonstrar usando filtros
 % a demonstração tem dois passos:
 
 % 1. any image of an ultrafilter is an ultrafilter 
 % (in particular, any projection from a product space)
 
 % -> esse ponto é a proposição do ultrafiltro push forward, então done
 
 % 2. any filter in the product space converges 
 % if and only if all its projections converge
 % -> esse é o teorema 3.3.2 do Cap 3
 
 Da Proposição \ref{prop:imag_ultrafiltro},
 sabemos que imagem de ultrafiltro é ultrafiltro,
 então em particular as imagens de ultrafiltros no
 espaço produto pelas projeções
 $\pi_{\alpha}$
 em cada
 $X_{\alpha}$
 são também ultrafiltros.
 
%  O Teorema \ref{teo:4.2} 
%  nos diz que os ultrafiltros
%  $(\pi_{\alpha})_*(\mathfrak{U})$
%  convergem para pelo menos um ponto em cada
%  $X_{\alpha}$
%  se, e somente se, o ultrafiltro
%  $\mathfrak{U}$
%  no espaço produto converge para pelo menos um ponto.
 
 Ora, então o Teorema \ref{teo:2.1}
 nos permite concluir que 
 $X$
 é compacto já que, por hipótese, os
 $X_{\alpha}$ 
 são compactos e, portanto, os
 $(\pi_{\alpha})_*(\mathfrak{U})$
 convergem para pelo menos um ponto pelo
 Teorema \ref{teo:4.2}.
 \end{proof}