\chapter{Introdução}

  Aqui eu estou escrevendo a introdução do trabalho.
  Você não precisa ter uma introdução.
  Pode apagar esse arquivo, se quiser.

  Quebre as linhas.
  O $\TeX$ só considera mudança de parágrafo quando você colocar uma linha em branco.
  Não escreva um linhão para o parágrafo inteiro que isso fica muito ruim de ler aqui no editor de texto. Se você escrever um montão de coisas, a linha fica muito longa e ruim de ler. Claro que no final, o resultado visual é o mesmo. Mas não faça isso! :-)

  Quando for escrever uma expressão matemática maior{\ldots}
  não escreva a expressão na mesma linha que o texto:
  $ax^2 + bx + c = 5$.

  Utilize as macros para $\reals$, $\complexes$, etc.
  Também tem umas macros boas pra funções
  \begin{equation*}
    \functionarray{f}{X}{Y}{x}{f(x)}.
  \end{equation*}
  E também para conjuntos{\ldots}
  tipo
  \begin{equation*}
    \Omega
    =
    \setsuchthat{x \in A}{\frac{\frac{x}{2} + \frac{x}{3}}{2} \geq 4}.
  \end{equation*}


  {\section{Primeira Seção}

  JAMAIS chame o seu arquivo de \textbf{010\_primeira\_secao.tex}!!!
  Seus arquivos devem ter um nome que represente o conteúdo.
}
  {\section{Segunda Seção}

  JAMAIS chame o seu arquivo de \textbf{020\_segunda\_secao.tex}!!!
  Seus arquivos devem ter um nome que represente o conteúdo.

  Se quiser,
  pode citar algum autor{\ldots}
  veja \cite[theorem 2.5]{walters}.
  Ou então,
  dê uma olhada em \cite[corolary 5.2]{misiurewicz}.
}
