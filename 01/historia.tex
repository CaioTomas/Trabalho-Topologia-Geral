\section{Sobre o teorema de Tychonoff}

 %falar da historia do teorema e aplicações
 %Wiki
 
 O teorema de Tychonoff diz que o produto de qualquer coleção
 de espaços topológicos compactos é compacto com respeito
 à topologia produto.
 O nome do teorema é uma homenagem ao matemático
 Andrey Nikolayevich Tikhonov 
 (cujo último nome é transcrito como Tychonoff),
 que mostrou esse resultado pela primeira vez em 1930
 para potências do invervalo 
 $[0,1]$
 e enunciou em 1935 o teorema na sua forma geral,
 junto com a observação de que a demonstração era idêntica
 à do caso especial.
 A primeira demonstração publicada da qual se tem conhecimento
 está em um artigo de 1937 escrito por Eduard Čech.
 
 O teorema de Tychonoff é considerado, junamente com o lema de Urysohn, 
 um dos resultados mais importantes em topologia geral.
 Ele depende crucialmente das definições precisa de compacidade
 e topologia produto; de fato, 
 o artigo de Tychonoff de 1935 define a topologia produto pela primeira vez.
 Por outro lado, parte da importância do teorema é dar confiança de que
 essas definições, em específico, são as mais úteis, ou mais
 ``bem comportadas''.
 
 De fato, a definição popular no século XIX e início do século XX era
 o critério de Bolzano-Weierstrass de que toda sequência admite uma
 subsequência convergente. Hoje, essa definição é a de 
 \textbf{compacidade sequencial}.

 É quase trivial mostrar que o produto de dois espaços sequencialmente
 compactos é sequencialmente compacto: basta passar de uma subsequência 
 do primeiro espaço e depois uma subsubsequência do segundo espaço.
 Um argumento por ``diagonalização'' um pouco mais elaborado dá conta
 do recado para um produto enumerável de espaços sequencialmente
 compactos. 
 Entretanto, o produto de uma quantidade não-enumerável de cópias
 do intervalo
 $[0,1]$
 (com a topologia usual) não é sequencialmente compacto com respeito
 à topologia produto, mesmo sendo compacto pelo teorema de Tychonoff.
 Essa falha tem relação com a compactificação de Stone-Čech, e o leitor
 interessado poderá saber mais em \cite{WikiTychonoff}.
 
 Ademais, o teorema de Tychonoff é usado na demonstração de vários outros
 teoremas, que incluem o teorema de Banach-Alaoglu, sobre a compacidade
 fraca-* da bola unitária do espaço dual de um espaço vetorial normado
 e o teorema de Arzelà-Ascoli, caracterizando as sequências de funções em
 que cada subsequência tem uma subsequência uniformemente convergente.
 Outros teoremas menos obviamente relacionados com compacidade mas que
 usam o teorema de Tychonoff são o teorema de De Bruijn-Erdős, que diz que
 todo grafo $k$-cromático minimal é finito, e o teorema de
 Curtis-Hedlund-Lyndon, que fornece uma caracterização topológica dos
 autômatos celulares. 
 Novamente, o leitor interessado é convidado a conferir \cite{WikiTychonoff}
 e as referências lá citadas para saber mais sobre cada um desses teoremas.
 É possível, ainda, mostrar o axioma da escolha assumindo o teorema de Tychonoff:
 uma prova é dada em \cite{WikiTychonoff}, também.
 
 Várias demonstrações diferentes já foram dadas para o teorema.
 A prova do próprio Tychonoff, de 1930, usava o conceito de um
 ponto de acumulação completo.
 
 Uma segunda maneira de demonstrar o teorema é usando filtros, 
 como um ``corolário'' do teorema da sub-base de Alexander. 
 Essa é a prova que apresentamos no texto.
 
 Uma terceira forma é usando redes universais, que é bastante análoga à
 demonstração com filtros. 
 Uma prova usando redes mas não redes universais foi dada em 1992 por
 Paul Chernoff.