\section{Background}
 
 Nesta subseção fornecemos as definições e noções básicas
 para a compreensão do texto, baseadas em \cite{caldas}.
 
 \begin{definition}[Espaço topológico]
 Seja
 $X$
 um conjunto qualquer.
 Dizemos que uma família
 $\tau_X \subset \parts X$
 define uma \textbf{topologia} em
 $X$,
 ou que
 $(X, \tau_X)$
 é um \textbf{espaço topológico}, quando
 $\tau_X$
 satisfaz:
 \begin{enumerate}
     \item $\varnothing, X \in \tau_X$;
     \item $A, B \in \tau_X \implies A \cap B \in \tau_X$ (fechado por interseção finita);
     \item $A_{\lambda} \in \tau_X, \lambda \in \Lambda 
     \implies \displaystyle{ \bigcup_{\lambda \in \Lambda} A_{\lambda} \in \tau_X }$ (fechado por união arbitrária).
 \end{enumerate}
 Os subconjuntos
 $A \subset X$
 tais que
 $A \in \tau_X$
 são chamados de \textbf{abertos} do espaço topológico
 $(X, \tau_X)$.
 Quando conveniente, por abuso de notação, diremos que
 $X$
 é um espaço topológico.
 \end{definition}
 
 \begin{definition}[Vizinhanças]
 Seja
 $(X, \tau_X)$
 um espaço topológico.
 Dado
 $x \in X$,
 uma \textbf{vizinhança aberta} de
 $x$
 é um aberto 
 $A \in \tau_X$
 que contém o ponto 
 $x$.
 Uma \textbf{vizinhança} de
 $x$
 é qualquer conjunto que contenha uma 
 vizinhança aberta de
 $x$.
 Denotamos por
 $\mathcal{V}(x)$
 a família de todas as vizinhanças de
 $x$.
 \end{definition}
 
 \begin{definition}[Fecho]
 Seja
 $(X, \tau_X)$
 um espaço topológico e
 $B \subset X$
 um subconjunto qualquer de
 $X$.
 Definimos o \textbf{fecho} de
 $B$, 
 denotado por
 $\closure B$,
 o conjunto
 \begin{equation*}
     \closure B
     =
     \setsuchthat{ x \in X }{ \forall V \in \mathcal{V}(x), V \cap B \neq \varnothing }.
 \end{equation*}
 Também escrevemos
 $\closureop (B)$
 ou, quando queremos enfatizar a topologia,
 $\closureop_{\tau_X}(B)$.
 \end{definition}
 
 \begin{definition}[Interior]
 Seja 
 $(X, \tau_X)$
 um espaço topológico e
 $B \subset X$
 um subconjunto qualquer de 
 $X$.
 O \textbf{interior} de 
 $B$
 é o maior conjunto aberto contido em
 $B$.
 Denotamos o interior de
 $B$
 por 
 $\interior B$
 ou ainda
 $\interiorop_{\tau_X}(B)$
 quando queremos dar ênfase ao fato de que o interior é tomado
 em relação à topologia
 $\tau_X$.
 \end{definition}
 
 \begin{definition}
 Se
 $\tau_1, \tau_2$
 são duas topologias em um mesmo conjunto
 $X$
 e
 $\tau_1 \subset \tau_2$,
 então dizemos que
 $\tau_2$
 é \textbf{mais forte} ou \textbf{mais fina} que
 $\tau_1$.
 Dizemos também que
 $\tau_1$
 é \textbf{mais fraca} que
 $\tau_2$.
 \end{definition}
 
 \begin{obs}
 A relação ``mais forte que'' define uma ordem parcial
 na família
 \begin{equation*}
     \mathcal{T}(X)
     =
     \setsuchthat{ \tau_X \subset \parts X }{ \tau_X \text{ é topologia} },
 \end{equation*}
 ou seja, a família das topologias em um conjunto
 $X$.
 
 Existe um elemento máximo dentre todas as topologias de 
 $X$:
 o conjunto das partes de
 $X$,
 que é a topologia mais forte que pode ser definida em
 $X$. 
 Por outro lado, 
 $\{ \varnothing, X \}$
 é a topologia mais fraca que pode ser definida em 
 $X$.
 
 É possível mostrar que a interseção
 $\tau_{\delta}$
 de uma família arbitrária
 $\tau_{\lambda}$
 de topologias é também uma topologia, e é a 
 \textbf{maior topologia que é menor que todas} as
 $\tau_{\lambda}$.
 A topologia 
 $\tau_{\delta}$
 é o \textbf{ínfimo} das
 $\tau_{\lambda}$.
 Escrevemos
 \begin{equation*}
     t_{\delta} 
     =
     \bigwedge \setsuchthat{ \tau_{\lambda} }{ \lambda \in \Lambda }
 \end{equation*}
 ou
 \begin{equation*}
     t_{\delta} 
     =
     \bigwedge_{\lambda \in \Lambda} \tau_{\lambda}.
 \end{equation*}
 Por outro lado, a união arbitrária de topologias nem sempre é uma topologia.
 Não obstante, se considerarmos a família
 \begin{equation*}
     \mathcal{F}
     =
     \setsuchthat{ \tau_X \in \mathcal{T}(X) }{ \bigcup_{\lambda \in \Lambda} \tau_{\lambda} \subset \tau_X },
 \end{equation*}
 ou seja, a família de todas as topologias que são
 maiores que todas as 
 $\tau_{\lambda}$,
 sabemos que
 $\mathcal{F} \neq \varnothing$,
 pois 
 $\parts X \in \mathcal{F}$.
 Seja então
 $\tau_{\sigma}$
 o ínfimo de
 $\mathcal{F}$:
 \begin{equation*}
     \tau_{\sigma}
     =
     \bigwedge \mathcal{F}.
 \end{equation*}
 A topologia
 $\tau_{\sigma}$
 é a menor topologia que é maior que é
 maior que todas as
 $\tau_{\lambda}$.
 Essa topologia é o supremo das
 $\tau_{\lambda}$,
 denotada por
 \begin{equation*}
     \tau_{\sigma}
     =
     \bigvee_{\lambda \in \Lambda} \tau_{\lambda}.
 \end{equation*}
 \end{obs}
 
 \begin{definition}[Topologia gerada]
 Sejam 
 $X$
 um conjunto qualquer e
 $\mathcal{C} \subset \parts X$
 uma família qualquer de subconjuntos de
 $X$.
 A topologia
 \begin{equation*}
     \tau(\mathcal{C}) = \bigvee_{\mathcal{C} \subset \tau_X} \tau_X
 \end{equation*}
 é a topologia gerada por 
 $\mathcal{C}$.
 Esas é a menor topologia de
 $X$
 que contém a família
 $\mathcal{C}$.
 \end{definition}
 
 \begin{definition}[Base]
 Seja
 $(X, \tau_X)$
 um espaço topológico.
 Uma família
 $\mathcal{B} \subset \tau_X$
 é uma \textbf{base} para a topologia
 $\tau_X$
 quando todo conjunto
 $A \in \tau_X$
 puder ser escrito como união de elementos de
 $\mathcal{B}$.
 Aqui seguimos a convenção de que
 \begin{equation*}
     \bigcup_{A \in \varnothing} A = \varnothing.
 \end{equation*}
 \end{definition}
 
 \begin{definition}[Sub-base]
 Seja
 $(X, \tau_X)$
 um espaço topológico. 
 Uma família
 $\mathcal{B}$
 de elementos de
 $\tau_X$
 é uma \textbf{sub-base} da topologia se
 $\mathcal{B}$
 gera 
 $\tau_X$, 
 ou seja,
 se
 $\tau_X$
 é a menor topologia contendo
 $\mathcal{B}$:
 qualquer topologia
 $\tau_X'$
 em 
 $X$
 contendo
 $\mathcal{B}$
 deve conter
 $\tau_X$
 também.
 \end{definition}
 
 \begin{definition}[Propriedade de interseção finita]
 Sejam
 $X$
 um conjunto e
 $\mathcal{A} = \{A_i\}_{i\in I}$
 uma família não vazia de subconjuntos de
 $X$
 indexada por 
 $I$.
 Dizemos que
 $\mathcal{A}$
 tem a \textbf{propriedade de interseção finita}
 se
 $\displaystyle{ \bigcap _{i\in J}A_{i} }$
 é não vazia para todo
 $J \subseteq I$
 não vazio.
 \end{definition}