\section{Outline do texto}

 %explicar cada seção brevemente, citando as referências
 
 O texto está dividido em 3 capítulos 
 (claro, sem contar esta introdução).
 
 O Capítulo 2 apresenta o conceito de
 filtros e ultrafiltros, demonstrando algumas
 propriedades básicas desses objetos.
 
 No Capítulo 3 falamos de topologia usando
 o ferramental dos filtros, caracterizando
 compacidade e a topologia de um espaço.
 As referências para esse capítulo foram \cite{caldas, videoultra, moorhouse}.
 
 No Capítulo 4, recordamos a caracterização de
 compacidade utilizando filtros, dada no capítulo
 anterior, e demonstramos o Teorema de Tychonoff e
 o Teorema da Sub-Base de Alexander. 
 Primeiro demonstramos o Teorema de Tychonoff 
 utilizando ultrafiltros; em seguida, 
 apresentamos a prova do Teorema da Sub-Base 
 (\cite{WikiAlexander})
 e, por fim, apresentamos uma demonstração
 alternativa do Teorema de Tychonoff, que
 utiliza o Teorema da Sub-Base (\cite{WikiTychonoff})
 
 O restante deste capítulo se baseia em 
 \cite{caldas, munkres} e se ocupa em dar 
 as definições básicas para o entendimento 
 do que será discutido nos capítulos que se seguirão.