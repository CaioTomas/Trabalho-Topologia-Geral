%
% Orientações:
%
% - Neste template... tudo o que vocẽ não souber para que serve...
%   pergunte!!! :-)
%
% - Utilize o "xelatex" e não o "pdflatex" para compilar.
% - Utilize o programa "bibtex" para gerar a bibliografia.
%   $ xelatex trabalho_final_de_topologia_geral
%   $ bibtex trabalho_final_de_topologia_geral
%   $ xelatex trabalho_final_de_topologia_geral
%   $ xelatex trabalho_final_de_topologia_geral
%
% - Coloque os dados bibliográficos no arquivo
%   trabalho_final_de_topologia_geral.bib
%
% - Não modifique os arquivos:
%   1. trabalho_final_de_topologia_geral.tex
%   2. definitions.tex
%
% - Crie macros! Coloque-as em:
%   definitions_local.tex
%
% - Se quiser incluir algum pacote (\usepackage{}),
%   faça-o no arquivo settings_local.tex
%
% - Divida o seu trabalho em "chapters", "sections" e "subsections".
%
% - Cada "chapter", "section" e "subsection" deve ficar em um arquivo separado.
%   Modifique o arquivo body.tex para incluir seus arquivos.
%
% - NÃO chame os arquivos de "sec1.tex", "sec2.tex" ou coisa do tipo.
%   Use um nome que indique seu conteúdo.
%
% - Você pode numerar os arquivos para que fiquem em ordem.
%   Por exemplo, "01_introducao.tex", "02_resultados_preliminares.tex", etc...
%
%
\documentclass[a4paper]{memoir}

\sloppy

\usepackage{fontspec}
\usepackage{xltxtra}
\usepackage{polyglossia}
\setdefaultlanguage{brazil}

\usepackage[none]{hyphenat}


\usepackage{amsmath}
\usepackage{amsthm}
\usepackage{amsfonts, amssymb}

\usepackage{graphicx}
\usepackage[textsize=footnotesize]{todonotes}
\usepackage[final]{hyperref}

% Utilizamos o pacote "memoir" para o layout do texto.
% Se quiser modificar o estilo do layout, faça-o aqui.

%\headstyles{xxx}
%\chapterstyle{pedersen}
\chapterstyle{madsen}


\author{Ponha seu nome}
\title{Ponha um título}

\RequirePackage{mathrsfs} % Font \mathscr

\theoremstyle{plain}

\newtheorem{theorem}{Teorema}[section]

\newtheorem{proposition}[theorem]{Proposição}
\newtheorem{corollary}[theorem]{Corolário}
\newtheorem{lemma}[theorem]{Lema}
\newtheorem*{notation}{Notação}
\newtheorem{definition}[theorem]{Definição}

% Horrível: o estilo "definition" não é bom para definição e notação!
% Horrível: estilo "remark" é ruim para observações e exemplos!!
\theoremstyle{definition}
\newtheorem{example}[theorem]{Exemplo}
\newtheorem{obs}[theorem]{Observação}


% Commonly used sets.
\providecommand{\naturals}[0]{\mathbb{N}}
\providecommand{\integers}[0]{\mathbb{Z}}
\providecommand{\rationals}[0]{\mathbb{Q}}
\providecommand{\reals}[0]{\mathbb{R}}
\providecommand{\complexes}[0]{\mathbb{C}}

\providecommand{\family}[1]{{\mathscr{#1}}}
\providecommand{\parts}[1]{{\family{P}\left(#1\right)}}


% Special symbols.
\providecommand{\suchthat}[0]{{\thickspace | \thickspace}}
\providecommand{\cardinality}[1]{{\# #1}}

% Brackets.
\newcommand{\threebracket}[5]{{\left #1 \vphantom{{#2}^N}\vphantom{{#4}^N}{#2} \, \right #3 \left. \vphantom{{#2}^N}\vphantom{{#4}^N}{#4} \right #5}}

\providecommand{\setsuchthat}[2]{{\threebracket{\{}{#1}{|\:}{#2}{\}}}}
\providecommand{\set}[1]{{\left \{ {#1} \right \}}}
\providecommand{\abs}[1]{{\left\lvert#1\right\rvert}}
\providecommand{\norm}[2][]{{\left\lVert{#2}\right\rVert_{#1}}}

\newcommand{\integral}[3][]{{\int_{#1} {#2}\, \mathrm{d}{#3}}}

% Operator names.
\DeclareMathOperator{\identity}{id}
\DeclareMathOperator{\interiorop}{int}
\DeclareMathOperator{\closureop}{cl}
\DeclareMathOperator{\supportop}{supp}
\DeclareMathOperator{\qtpop}{qtp}

\DeclareMathOperator{\eigenspace}{eig}
\DeclareMathOperator{\expbase}{e}


% Functions
\newcommand{\function}[3]{{#1: #2 \to #3}}
\newcommand{\functionmaps}[3]{{#1: #2 \mapsto #3}}
\newcommand{\functionarray}[5]{{\begin{array}[t]{lrcl}
                                  #1:& #2 &\to     &#3 \\
                                     & #4 &\mapsto &#5
                                \end{array}}}

% Operations
\newcommand{\closure}[1]{{\overline{#1}}}
\newcommand{\interior}[1]{\interiorop \left({#1}\right)}
\newcommand{\complementset}[1]{{#1}^{c}}
\newcommand{\powerset}[1]{{\mathscr{P}({#1})}}
\newcommand{\indicator}[1]{{I_{#1}}}
\newcommand{\support}[1]{{\supportop \left( {#1} \right)}}


\newcommand{\ball}[3][]{{B_{#1}\left({#3}; {#2}\right)}}
\newcommand{\balls}[2]{{\family{B}_{#1}\left({#2}\right)}}

%
% Coloque suas definições aqui.
%


%
% The headers will not be upper cased.
%
\nouppercaseheads


\begin{document}
  \frontmatter

  \maketitle
  \thispagestyle{empty}
  \clearpage

  \tableofcontents

  %Memoir page style for mainmatter.
  \pagestyle{ruled}
  \mainmatter

  {%
% Remova os meus e coloque os seus!
%

{\chapter{Introdução}
 
 \todo{Acho que a única coisa que mudou nesse
 capítulo desde a última vez que o senhor viu
 foi que eu escrevi o outline :-)}

 {\section{Sobre o teorema de Tychonoff}

 %falar da historia do teorema e aplicações
 %Wiki
 
 O teorema de Tychonoff diz que o produto de qualquer coleção
 de espaços topológicos compactos é compacto com respeito
 à topologia produto.
 O nome do teorema é uma homenagem ao matemático
 Andrey Nikolayevich Tikhonov 
 (cujo último nome é transcrito como Tychonoff),
 que mostrou esse resultado pela primeira vez em 1930
 para potências do invervalo 
 $[0,1]$
 e enunciou em 1935 o teorema na sua forma geral,
 junto com a observação de que a demonstração era idêntica
 à do caso especial.
 A primeira demonstração publicada da qual se tem conhecimento
 está em um artigo de 1937 escrito por Eduard Čech.
 
 O teorema de Tychonoff é considerado, junamente com o lema de Urysohn, 
 um dos resultados mais importantes em topologia geral.
 Ele depende crucialmente das definições precisa de compacidade
 e topologia produto; de fato, 
 o artigo de Tychonoff de 1935 define a topologia produto pela primeira vez.
 Por outro lado, parte da importância do teorema é dar confiança de que
 essas definições, em específico, são as mais úteis, ou mais
 ``bem comportadas''.
 
 De fato, a definição popular no século XIX e início do século XX era
 o critério de Bolzano-Weierstrass de que toda sequência admite uma
 subsequência convergente. Hoje, essa definição é a de 
 \textbf{compacidade sequencial}.

 É quase trivial mostrar que o produto de dois espaços sequencialmente
 compactos é sequencialmente compacto: basta passar de uma subsequência 
 do primeiro espaço e depois uma subsubsequência do segundo espaço.
 Um argumento por ``diagonalização'' um pouco mais elaborado dá conta
 do recado para um produto enumerável de espaços sequencialmente
 compactos. 
 Entretanto, o produto de uma quantidade não-enumerável de cópias
 do intervalo
 $[0,1]$
 (com a topologia usual) não é sequencialmente compacto com respeito
 à topologia produto, mesmo sendo compacto pelo teorema de Tychonoff.
 Essa falha tem relação com a compactificação de Stone-Čech, e o leitor
 interessado poderá saber mais em \cite{WikiTychonoff}.
 
 Ademais, o teorema de Tychonoff é usado na demonstração de vários outros
 teoremas, que incluem o teorema de Banach-Alaoglu, sobre a compacidade
 fraca-* da bola unitária do espaço dual de um espaço vetorial normado
 e o teorema de Arzelà-Ascoli, caracterizando as sequências de funções em
 que cada subsequência tem uma subsequência uniformemente convergente.
 Outros teoremas menos obviamente relacionados com compacidade mas que
 usam o teorema de Tychonoff são o teorema de De Bruijn-Erdős, que diz que
 todo grafo $k$-cromático minimal é finito, e o teorema de
 Curtis-Hedlund-Lyndon, que fornece uma caracterização topológica dos
 autômatos celulares. 
 Novamente, o leitor interessado é convidado a conferir \cite{WikiTychonoff}
 e as referências lá citadas para saber mais sobre cada um desses teoremas.
 É possível, ainda, mostrar o axioma da escolha assumindo o teorema de Tychonoff:
 uma prova é dada em \cite{WikiTychonoff}, também.
 
 Várias demonstrações diferentes já foram dadas para o teorema.
 A prova do próprio Tychonoff, de 1930, usava o conceito de um
 ponto de acumulação completo.
 
 Uma segunda maneira de demonstrar o teorema é usando filtros, 
 como um ``corolário'' do teorema da sub-base de Alexander. 
 Essa é a prova que apresentamos no texto.
 
 Uma terceira forma é usando redes universais, que é bastante análoga à
 demonstração com filtros. 
 Uma prova usando redes mas não redes universais foi dada em 1992 por
 Paul Chernoff.}
  
 {\section{Outline do texto}

 %explicar cada seção brevemente, citando as referências
 
 O texto está dividido em 3 capítulos 
 (claro, sem contar esta introdução).
 
 O Capítulo 2 apresenta o conceito de
 filtros e ultrafiltros, demonstrando algumas
 propriedades básicas desses objetos.
 
 No Capítulo 3 falamos de topologia usando
 o ferramental dos filtros, caracterizando
 compacidade e a topologia de um espaço.
 As referências para esse capítulo foram \cite{caldas, videoultra, moorhouse}.
 
 No Capítulo 4, recordamos a caracterização de
 compacidade utilizando filtros, dada no capítulo
 anterior, e demonstramos o Teorema de Tychonoff e
 o Teorema da Sub-Base de Alexander. 
 Primeiro demonstramos o Teorema de Tychonoff 
 utilizando ultrafiltros; em seguida, 
 apresentamos a prova do Teorema da Sub-Base 
 (\cite{WikiAlexander})
 e, por fim, apresentamos uma demonstração
 alternativa do Teorema de Tychonoff, que
 utiliza o Teorema da Sub-Base (\cite{WikiTychonoff})
 
 O restante deste capítulo se baseia em 
 \cite{caldas, munkres} e se ocupa em dar 
 as definições básicas para o entendimento 
 do que será discutido nos capítulos que se seguirão.}
  
 {\section{Background}
 
 Nesta subseção fornecemos as definições e noções básicas
 para a compreensão do texto, baseadas em \cite{caldas}.
 
 \begin{definition}[Espaço topológico]
 Seja
 $X$
 um conjunto qualquer.
 Dizemos que uma família
 $\tau_X \subset \parts X$
 define uma \textbf{topologia} em
 $X$,
 ou que
 $(X, \tau_X)$
 é um \textbf{espaço topológico}, quando
 $\tau_X$
 satisfaz:
 \begin{enumerate}
     \item $\varnothing, X \in \tau_X$;
     \item $A, B \in \tau_X \implies A \cap B \in \tau_X$ (fechado por interseção finita);
     \item $A_{\lambda} \in \tau_X, \lambda \in \Lambda 
     \implies \displaystyle{ \bigcup_{\lambda \in \Lambda} A_{\lambda} \in \tau_X }$ (fechado por união arbitrária).
 \end{enumerate}
 Os subconjuntos
 $A \subset X$
 tais que
 $A \in \tau_X$
 são chamados de \textbf{abertos} do espaço topológico
 $(X, \tau_X)$.
 Quando conveniente, por abuso de notação, diremos que
 $X$
 é um espaço topológico.
 \end{definition}
 
 \begin{definition}[Vizinhanças]
 Seja
 $(X, \tau_X)$
 um espaço topológico.
 Dado
 $x \in X$,
 uma \textbf{vizinhança aberta} de
 $x$
 é um aberto 
 $A \in \tau_X$
 que contém o ponto 
 $x$.
 Uma \textbf{vizinhança} de
 $x$
 é qualquer conjunto que contenha uma 
 vizinhança aberta de
 $x$.
 Denotamos por
 $\mathcal{V}(x)$
 a família de todas as vizinhanças de
 $x$.
 \end{definition}
 
 \begin{definition}[Fecho]
 Seja
 $(X, \tau_X)$
 um espaço topológico e
 $B \subset X$
 um subconjunto qualquer de
 $X$.
 Definimos o \textbf{fecho} de
 $B$, 
 denotado por
 $\closure B$,
 o conjunto
 \begin{equation*}
     \closure B
     =
     \setsuchthat{ x \in X }{ \forall V \in \mathcal{V}(x), V \cap B \neq \varnothing }.
 \end{equation*}
 Também escrevemos
 $\closureop (B)$
 ou, quando queremos enfatizar a topologia,
 $\closureop_{\tau_X}(B)$.
 \end{definition}
 
 \begin{definition}[Interior]
 Seja 
 $(X, \tau_X)$
 um espaço topológico e
 $B \subset X$
 um subconjunto qualquer de 
 $X$.
 O \textbf{interior} de 
 $B$
 é o maior conjunto aberto contido em
 $B$.
 Denotamos o interior de
 $B$
 por 
 $\interior B$
 ou ainda
 $\interiorop_{\tau_X}(B)$
 quando queremos dar ênfase ao fato de que o interior é tomado
 em relação à topologia
 $\tau_X$.
 \end{definition}
 
 \begin{definition}
 Se
 $\tau_1, \tau_2$
 são duas topologias em um mesmo conjunto
 $X$
 e
 $\tau_1 \subset \tau_2$,
 então dizemos que
 $\tau_2$
 é \textbf{mais forte} ou \textbf{mais fina} que
 $\tau_1$.
 Dizemos também que
 $\tau_1$
 é \textbf{mais fraca} que
 $\tau_2$.
 \end{definition}
 
 \begin{obs}
 A relação ``mais forte que'' define uma ordem parcial
 na família
 \begin{equation*}
     \mathcal{T}(X)
     =
     \setsuchthat{ \tau_X \subset \parts X }{ \tau_X \text{ é topologia} },
 \end{equation*}
 ou seja, a família das topologias em um conjunto
 $X$.
 
 Existe um elemento máximo dentre todas as topologias de 
 $X$:
 o conjunto das partes de
 $X$,
 que é a topologia mais forte que pode ser definida em
 $X$. 
 Por outro lado, 
 $\{ \varnothing, X \}$
 é a topologia mais fraca que pode ser definida em 
 $X$.
 
 É possível mostrar que a interseção
 $\tau_{\delta}$
 de uma família arbitrária
 $\tau_{\lambda}$
 de topologias é também uma topologia, e é a 
 \textbf{maior topologia que é menor que todas} as
 $\tau_{\lambda}$.
 A topologia 
 $\tau_{\delta}$
 é o \textbf{ínfimo} das
 $\tau_{\lambda}$.
 Escrevemos
 \begin{equation*}
     t_{\delta} 
     =
     \bigwedge \setsuchthat{ \tau_{\lambda} }{ \lambda \in \Lambda }
 \end{equation*}
 ou
 \begin{equation*}
     t_{\delta} 
     =
     \bigwedge_{\lambda \in \Lambda} \tau_{\lambda}.
 \end{equation*}
 Por outro lado, a união arbitrária de topologias nem sempre é uma topologia.
 Não obstante, se considerarmos a família
 \begin{equation*}
     \mathcal{F}
     =
     \setsuchthat{ \tau_X \in \mathcal{T}(X) }{ \bigcup_{\lambda \in \Lambda} \tau_{\lambda} \subset \tau_X },
 \end{equation*}
 ou seja, a família de todas as topologias que são
 maiores que todas as 
 $\tau_{\lambda}$,
 sabemos que
 $\mathcal{F} \neq \varnothing$,
 pois 
 $\parts X \in \mathcal{F}$.
 Seja então
 $\tau_{\sigma}$
 o ínfimo de
 $\mathcal{F}$:
 \begin{equation*}
     \tau_{\sigma}
     =
     \bigwedge \mathcal{F}.
 \end{equation*}
 A topologia
 $\tau_{\sigma}$
 é a menor topologia que é maior que é
 maior que todas as
 $\tau_{\lambda}$.
 Essa topologia é o supremo das
 $\tau_{\lambda}$,
 denotada por
 \begin{equation*}
     \tau_{\sigma}
     =
     \bigvee_{\lambda \in \Lambda} \tau_{\lambda}.
 \end{equation*}
 \end{obs}
 
 \begin{definition}[Topologia gerada]
 Sejam 
 $X$
 um conjunto qualquer e
 $\mathcal{C} \subset \parts X$
 uma família qualquer de subconjuntos de
 $X$.
 A topologia
 \begin{equation*}
     \tau(\mathcal{C}) = \bigvee_{\mathcal{C} \subset \tau_X} \tau_X
 \end{equation*}
 é a topologia gerada por 
 $\mathcal{C}$.
 Esas é a menor topologia de
 $X$
 que contém a família
 $\mathcal{C}$.
 \end{definition}
 
 \begin{definition}[Base]
 Seja
 $(X, \tau_X)$
 um espaço topológico.
 Uma família
 $\mathcal{B} \subset \tau_X$
 é uma \textbf{base} para a topologia
 $\tau_X$
 quando todo conjunto
 $A \in \tau_X$
 puder ser escrito como união de elementos de
 $\mathcal{B}$.
 Aqui seguimos a convenção de que
 \begin{equation*}
     \bigcup_{A \in \varnothing} A = \varnothing.
 \end{equation*}
 \end{definition}
 
 \begin{definition}[Sub-base]
 Seja
 $(X, \tau_X)$
 um espaço topológico. 
 Uma família
 $\mathcal{B}$
 de elementos de
 $\tau_X$
 é uma \textbf{sub-base} da topologia se
 $\mathcal{B}$
 gera 
 $\tau_X$, 
 ou seja,
 se
 $\tau_X$
 é a menor topologia contendo
 $\mathcal{B}$:
 qualquer topologia
 $\tau_X'$
 em 
 $X$
 contendo
 $\mathcal{B}$
 deve conter
 $\tau_X$
 também.
 \end{definition}
 
 \begin{definition}[Propriedade de interseção finita]
 Sejam
 $X$
 um conjunto e
 $\mathcal{A} = \{A_i\}_{i\in I}$
 uma família não vazia de subconjuntos de
 $X$
 indexada por 
 $I$.
 Dizemos que
 $\mathcal{A}$
 tem a \textbf{propriedade de interseção finita}
 se
 $\displaystyle{ \bigcap _{i\in J}A_{i} }$
 é não vazia para todo
 $J \subseteq I$
 não vazio.
 \end{definition}}
  
  
   }

{\chapter{Filtros e ultrafiltros}

{\section{Definições e propriedades básicas}
\label{sec:propriedades}

 \todo{Nessa seção eu acrescentei as propriedades
 que o senhor havia mencionado e mudei a
 definição de ultrafiltro, prof.
 Além disso, para mostrar que toda família
 com a propriedade da interseção finita
 está em um ultrafiltro eu mostrei primeiro
 o Lema do Ultrafiltro.}

 %Podemos dar uma definição alternativa de vizinhanças
 %baseada na operação de interior.

%  \begin{definition}[Vizinhanças]
%  Dado um conjunto $X$ e $a \in X$, dizemos que 
%  $\mathcal{V}(a)$ 
%  é um conjunto de vizinhanças de $a$ se
 
%  \begin{enumerate}
%      \item $V \in \mathcal{V}(a) \implies a\in V$;
%      \item $V\in \mathcal{V}(a),
%      V \subset W \implies W \in \mathcal{V}(a)$;
%      \item $X \in \mathcal{V}(a)$;
%      \item $V, W \in \mathcal{V}(a) 
%      \implies V \cap W \in \mathcal{V}(a)$;
%      \item $V \in \mathcal{V}(a) 
%      \implies \interiorop (V) \in \mathcal{V}(a)$.
%  \end{enumerate}
 
%  \end{definition}

 Uma primeira observação interessante de se fazer
 é que a família de vizinhanças de 
 $a \in X$,
 $\mathcal{V}(a)$,
 é um filtro que tem
 $\tau (a)$
 como base, como mostraremos mais à frente.

 \begin{definition}[Filtro]
 Dado um conjunto $X$, dizemos que 
 $\mathfrak{F} \subset \powerset X$ 
 é um \textbf{filtro} em $X$ 
 se as seguintes propriedades são satisfeitas:

 \begin{enumerate}
    \item $\mathfrak{F} \neq \varnothing$;
    \item $\varnothing \notin \mathfrak{F}$;
    \item $V, W \in \mathfrak{F} 
    \implies V\cap W \in \mathfrak{F}$;
    \item $V \in \mathfrak{F}, V \subset W
    \implies W \in \mathfrak{F}$.
 \end{enumerate}

 \end{definition}
 
 \begin{proposition}
 O conjunto das vizinhanças de $a\in X$ é um filtro.
 \end{proposition}
 
 \begin{proof}
 Por definição, 
 $X \in \mathcal{V}(a)$,
 de modo que 
 $\mathcal{V}(a) \neq \varnothing$.
 Ademais, como
 $a\in V, \forall V \in \mathcal{V}(a)$,
 então nenhum elemento de 
 $\mathcal{V}(a)$
 é vazio, ou seja,
 $\varnothing \notin \mathcal{V}(a)$.
 A terceira e quarta propriedades dos filtros seguem
 diretamente da quarta e da segunda propriedade das
 vizinhanças, respectivamente.
 \end{proof}
 
 É importante notar que
 
 \begin{proposition}
 Dois conjuntos disjuntos não podem estar
 no mesmo filtro.
 \end{proposition}
 
 \begin{proof}
 Sejam 
 $A, B \subset X$
 disjuntos e
 $\mathfrak{F}$
 um filtro em 
 $X$.
 Suponha que
 $A, B \in \mathfrak{F}$.
 Ora, então por definição teríamos
 $A \cap B = \varnothing \in \mathfrak{F}$,
 absurdo.
 \end{proof}
 
 Consequentemente, como a definição a seguir mostrará,
 também não podemos ter conjuntos disjuntos no mesmo
 ultrafiltro.
 
 \begin{definition}[Ultrafiltro]
 Um \textbf{ultrafiltro} em
 $X$
 é um filtro
 $\mathfrak{U}$
 tal que os únicos filtros que o contêm são
 $\powerset X$
 e
 $\mathfrak{U}$.
 Dito de outro modo, um ultrafiltro é
 um \textbf{filtro maximal}.
 \end{definition}
 
 Usando a proposição anterior,
 uma maneira alternativa de enunciar a maximalidade,
 que nos dá uma outra definição clássica de ultrafiltro é
 ``um ultrafiltro em 
 $X$
 é um filtro
 $\mathfrak{U}$
 tal que para todo
 $Y \subseteq X$,
 ou
 $Y \in \mathfrak{U}$
 ou
 $X \setminus Y \in \mathfrak{U}$,
 não podendo acontecer ambos.''
 
 Ainda usando a proposição anterior, 
 temos a seguinte propriedade.
 
 \begin{proposition}
 Um ultrafiltro
 $\mathfrak{U}$
 em 
 $X$
 satisfaz a seguinte propriedade:
 dada uma partição qualquer
 \begin{equation*}
    X = X_1 \sqcup X_2
 \end{equation*}
 de $X$ em dois subconjuntos, 
 apenas um dos $X_i$ pertence a $\mathfrak{U}$. 
 Aqui,
 $\sqcup$
 denota união disjunta.
 \end{proposition}
 
 \begin{proof}
 Segue do fato de que conjuntos disjuntos não podem pertencer ao mesmo filtro.
 \end{proof}
 
 Interessantemente, não precisamos que
 $A$
 e
 $B$
 na proposição acima sejam disjuntos para
 concluirmos que um deles está no ultrafiltro
 (mas não concluiremos que \textbf{apenas um}
 deles está no filtro).
 De fato, melhor e mais útil é a proposição
 a seguir.
 
 \begin{proposition}
 Se 
 $\mathfrak{U}$ 
 é um ultrafiltro e 
 $X = A_1 \cup \dotsb \cup A_n$, 
 então existe
 $j$
 tal que 
 $A_j \in \mathfrak{U}$.
 \end{proposition}
 
 \begin{proof}
 Se nenhum dos
 $A_j$
 estivesse no ultrafiltro,
 então o complementar da união estaria em
 $\mathfrak{U}$.
 Ora, mas o complementar da união é
 $\varnothing$,
 absurdo.
 \end{proof}
 
 % 0. Toda família com a propriedade de 
 % interseção finita está contida em um ultrafiltro. 
 % (você usa sempre que diz "estenda isso para um
 % ultrafiltro que contém...")
 
 O lema a seguir é frequentemente chamado de
 \textbf{Lema do Ultrafiltro}, do inglês
 \textit{Ultrafilter Lemma}, e sua demonstração
 utiliza o axioma da escolha na forma do
 lema de Zorn.
 
 \begin{lemma}[Ultrafiltro]
 Todo filtro em
 $X$
 está contido em um ultrafiltro (em $X$).
 \end{lemma}
 % encontrei essa demonstração
 % em https://proofwiki.org/wiki/Ultrafilter_Lemma,
 % mas não estou muito confortável com ela...
 % a afirmação que a interseção está na união
 % eu que fiz, porque não vi sentido no que o
 % site fez nessa parte
 \begin{proof}
 Seja
 $\Omega$
 o conjunto dos filtros em
 $X$.
 A relação de ordem parcial
 $\subseteq$
 ordena
 $\Omega$
 parcialmente. 
 Tomando uma cadeia não vazia
 $\mathscr{C} \subseteq \Omega$,
 isto é, um subconjunto totalmente ordenado de
 $\Omega$,
 temos que
 $\displaystyle{ \bigcup \mathscr{C} } \in \Omega$
 e, portanto,
 $\displaystyle{ \bigcup \mathscr{C} }$
 é uma cota superior de
 $\mathscr{C}$.
 De fato, se
 $A,B \in \displaystyle{ \bigcup \mathscr{C} }$
 então existem filtros
 $\mathfrak{F}, \mathfrak{F}' \in \mathscr{C}$
 tais que
 $A \in \mathfrak{F}$
 e
 $B \in \mathfrak{F}'$.
 Como 
 $\mathscr{C}$
 é uma cadeia, podemos assumir s.p.g. que
 $\mathfrak{F} \subseteq \mathfrak{F}'$.
 Logo,
 $A \in \mathfrak{F}'$
 e, daí,
 $A \cap B \in \mathfrak{F}'$.
 Em particular,
 $A \cap B \in \displaystyle{ \bigcup \mathscr{C} }$.
 Pelo Lema de Zorn, 
 para todo
 $\mathfrak{F} \in \Omega$
 existe um elemento maximal 
 (com respeito à relação $\subseteq$)
 $\mathfrak{F}'$
 tal que
 $\mathfrak{F} \subseteq \mathfrak{F}'$.
 Ora, mas então por definição temos
 $\mathfrak{F}'$
 ultrafiltro, como desejado.
 \end{proof}
 
 \begin{corollary}
 Toda família com a propriedade de interseção finita
 está contida em um ultrafiltro.
 \end{corollary}

 \begin{proof}
 Seja 
 $X$
 um conjunto não vazio e
 $\mathcal{A} \subset \powerset X$
 uma família com a propriedade da interseção finita.
 Pelo Lema do Ultrafiltro,
 existe um ultrafiltro
 $\mathfrak{U}$
 em 
 $X$
 tal que
 $\mathcal{A} \subseteq \mathfrak{U}$.
 \end{proof}
 
 Sempre que dissermos
 ``estenda para um ultrafiltro que contém...'',
 esse corolário está sendo utilizado,
 ainda que implicitamente.
 Ademais, essa proposição e a propriedade de que
 conjuntos disjuntos não pertencem ao mesmo
 filtro tornam a caracterização de Hausdorff
 imediata!}}

{\chapter{Topologia geral com filtros}

 {\section{Caracterização da topologia por ultrafiltros}
\label{sec:topol_filtros}
%Seção 2 da ref

 Seja 
 $X$
 um espaço topológico e 
 $\mathfrak{U}$
 um ultrafiltro em 
 $X$.
 Dizemos que 
 $\mathfrak{U}$
 \textbf{converge} para um ponto
 $x \in X$
 e denotamos
 $\mathfrak{U} \to x$
 se toda vizinhança aberta
 $U$
 de
 $x$
 é tal que
 $U \in \mathfrak{U}$.
 
 \begin{theorem}
 \label{teo:2.1}
 Seja
 $X$
 um espaço topológico.
 Então
 \begin{enumerate}[(a)]
    \item $X$ é Hausdorff se, e somente se, 
    todo ultrafiltro em $X$ converge para \textbf{no máximo} um ponto.
    \item $X$ é compacto se, e somente se, 
    todo ultrafiltro em $X$ converge para \textbf{pele menos} um ponto.
 \end{enumerate}
 \end{theorem}

 \begin{proof}
 \begin{enumerate}[(a)]
     \item Seja 
     $X$
     Hausdorff e suponha que
     $\mathfrak{U} \to x$
     e
     $\mathfrak{U} \to y$
     para dois pontos
     $x \neq y$.
     Sejam
     $U, V \subseteq X$
     vizinhanças abertas disjuntas de
     $x$
     e
     $y$,
     respectivamente. 
     Então temos
     $U, V \in \mathfrak{U}$
     e, portanto, 
     $\varnothing = U \cap V \in \mathfrak{U}$,
     absurdo.
     Logo,
     $\mathfrak{U}$
     converge para no máximo um ponto.
     
     Reciprocamente, suponha que todo ultrafiltro em
     $X$
     converge para no máximo um ponto, 
     e sejam
     $x \neq y$
     dois pontos de 
     $X$.
     Suponha que toda vizinhança aberta de 
     $x$
     tem interseção com toda vizinhança aberta de
     $y$.
     Então a família
     \begin{equation*}
         \mathfrak{G}
         =
         \setsuchthat{ \text{aberto } U \subseteq X }{ x \in U \text{ ou } y \in U }
     \end{equation*}
     tem a propriedade da interseção finita, de modo que
     $\mathfrak{G} \subseteq \mathfrak{U}$
     para algum ultrafiltro
     $\mathfrak{U}$.
     Se
     $U, V \subseteq X$
     são vizinhanças abertas de
     $x$
     e
     $y$, 
     respectivamente, então
     $U, V \in \mathfrak{U}$,
     de modo que
     $U \cap V \neq \varnothing$.
     
     \item Seja 
     $X$
     compacto e suponha que
     $\mathfrak{U}$
     é um ultrafiltro em
     $X$
     que não converge para nenhum ponto de
     $X$.
     Então para todo ponto
     $x \in X$,
     podemos encontrar uma vizinhança aberta
     $U_x$
     de
     $x$
     tal que
     $U_x \notin \mathfrak{U}$,
     ou seja,
     $X \setminus U_x \in \mathfrak{U}$.
     Obtemos então uma família de fechados
     \begin{equation*}
         \setsuchthat{ X \setminus U_x }{ x \in X }
     \end{equation*}
     com interseção vazia, já que
     $x \notin X \setminus U_x$.
     Daí, existem 
     $x_1, x_2, \dots, x_n \in X$
     tais que
     \begin{equation*}
         (X \setminus U_{x_1}) \cap \cdots \cap (X \setminus U_{x_2}) 
         = 
         \varnothing,
     \end{equation*}
     o que é impossível já que 
     $X \setminus U_{x_i} \in \mathfrak{U}$
     para todo 
     $i$.
     
     Reciprocamente, suponha que todo ultrafiltro em 
     $X$
     converge e suponha que
     $X$
     tem uma cobertura aberta
     $\mathcal{O}$
     que não possui subcobertura finita.
     Isso implica que a família de fechados
     \begin{equation*}
         \mathfrak{G}
         =
         \setsuchthat{ X \setminus U }{ U \in \mathcal{O} }
     \end{equation*}
     tem a propriedade da interseção finita.
     Estenda isso para um ultrafiltro
     $\mathfrak{U} \supseteq \mathfrak{G}$.
     Por hipótese, existe um ponto
     $x \in X$
     tal que 
     $\mathfrak{U} \to x$.
     O ponto 
     $x$
     é coberto por algum
     $U \in \mathcal{O}$,
     de modo que
     $U \in \mathfrak{U}$.
     Mas também temos
     $X \setminus U \in \mathfrak{G} \subseteq \mathfrak{U}$,
     de modo que
     $\mathfrak{U}$
     contém
     $(X \setminus U) \cap U = \varnothing$,
     uma contradição.
 \end{enumerate}
 \end{proof}
 
 Uma consequência do item (b) é que ``todo ultrafiltro em 
 $X$ 
 converge para pelo menos um ponto'' é equivalente a
 ``toda cobertura aberta de 
 $X$ 
 possui subcobertura finita''. 
 Portanto, definir compacidade como
 ``toda cobertura tem subcobertura finita''
 é equivalente a definir 
 ``todo ultrafiltro converge para pelo menos um ponto''.
 
 Ademais, note que o teorema acima nos diz que 
 $X$
 é Hausdorff e compacto se, e somente se, todo ultrafiltro em
 $X$
 converge para exatamente um ponto.
 Temos também a seguinte caracterização da topologia por ultrafiltros.

 %talvez tirar esse teorema
 \begin{theorem}
 \label{teo:2.2}
 Seja
 $X$
 um espaço topológico.
 $U \subseteq X$ 
 é aberto se, e somente se, 
 $U \in \mathfrak{U}$
 para todo ultrafiltro
 $\mathfrak{U}$
 convergindo para algum ponto de
 $U$.
 \end{theorem}
 \begin{proof}
 Se 
 $U \subseteq X$
 é aberto e
 $\mathfrak{U}$
 é um ultrafiltro convergindo para um ponto
 $u \in U$,
 então devemos ter
 $U \in \mathfrak{U}$
 pela definição da convergência
 $\mathfrak{U} \to u$.
 
 Reciprocamente, seja
 $U \subseteq X$
 e suponha que
 $U \in \mathfrak{U}$
 para todo ultrafiltro
 $\mathfrak{U}$
 convergindo para algum ponto de
 $U$.
 Se 
 $U$ 
 não é aberto, então existe
 $u \in U$
 tal que toda vizinhança aberta de 
 $u$
 contém pelo menos um ponto de
 $X \setminus U$.
 Isso implica que a coleção
 \begin{equation*}
     \mathfrak{G}
     =
     \left\{ X \setminus U \right\} \cup \left\{ \text{vizinhanças abertas de } u \right\}
 \end{equation*}
 tem a propriedade da interseção finita.
 Estenda essa família para um ultrafiltro
 $\mathfrak{U} \supseteq \mathfrak{G}$.
 Por construção, 
 $\mathfrak{U}$
 contém toda vizinhança aberta de 
 $u$,
 de modo que
 $\mathfrak{U} \to u$.
 Por hipótese, 
 $U \in \mathfrak{U}$,
 o que implica
 $\varnothing = (X \setminus U) \cap U \in \mathfrak{U}$,
 absurdo. 
 Logo, 
 $U$
 deve ser aberto, como afirmado.
 \end{proof}}
 
 {\section{Ultrafiltros \textit{push forward}}
\label{sec:push_forward}
%Prop 3.1, Teo 3.2 ref

 Nesta seção, introduzimos o \textit{push forward}
 de um ultrafiltro e provamos dois resultados sobre ele.
 \begin{proposition}
 \label{prop:imag_ultrafiltro}
 Sejam 
 $f:X \to Y$
 uma função qualquer entre conjuntos e
 $\mathfrak{U}$
 um ultrafiltro em $X$. Defina
 \begin{equation*}
     f_{*}(\mathfrak{U})
     =
     \setsuchthat{V \subseteq Y}{f^{-1}(V) \in \mathfrak{U}}.
 \end{equation*}
 Afirmamos que 
 $f_{*}(\mathfrak{U})$
 é um ultrafiltro em $Y$.
 \end{proposition}
 
 \begin{proof}
 Como
 $f^{-1}(\varnothing) = \varnothing \notin \mathfrak{U}$
 e
 $f^{-1}(Y) = X \in \mathfrak{U}$,
 temos que
 $\varnothing \notin f_*(\mathfrak{U})$
 e
 $Y \in f_*(\mathfrak{U})$.
 Se 
 $V_1, V_2 \in f_*(\mathfrak{U})$,
 então
 $f^{-1}(V_1 \cap V_2) = f^{-1}(V_1) \cap f^{-1}(V_2) \in \mathfrak{U}$,
 de modo que
 $V_1 \cap V_2 \in f_*(\mathfrak{U})$.
 Ademais, se
 $V_1 \in f_*(\mathfrak{U})$
 e
 $V_1 \subseteq V \subseteq Y$,
 então
 $f^{-1}(V) \supseteq f^{-1}(V_1) \in \mathfrak{U}$,
 de modo que
 $f^{-1}(V) \in \mathfrak{U}$
 e
 $V \in f_*(\mathfrak{U})$.
 Portanto, 
 $f_*(\mathfrak{U})$
 é um filtro em 
 $Y$. 
 Por fim, se
 $Y = Y_1 \sqcup Y_2$,
 então
 $X = f^{-1}(Y_1) \sqcup f^{-1}(Y_2)$,
 logo
 $f^{-1}(Y_i) \in \mathfrak{U}$
 para exatamente um dos
 $i$'s,
 nos dando
 $Y_i \in f_*(\mathfrak{U})$.
 Portanto,
 $f_*(\mathfrak{U})$
 é, de fato, um ultrafiltro em
 $Y$.
 \end{proof}
 
 O ultrafiltro
 $f_*(\mathfrak{U})$ 
 é o \textit{push forward} do ultrafiltro
 $\mathfrak{U}$,
 também chamado de \textbf{ultrafiltro imagem}.
 Essencialmente, essa proposição nos diz que
 imagem de ultrafiltro é também ultrafiltro 
 (e, na verdade, é possível mostrar ainda que
 imagem de filtro é também filtro).
 Note que se
 $f : X \to Y$
 e
 $g : Y \to Z$,
 então
 $(g \circ f)_*(\mathfrak{U}) = g_*(f_*(\mathfrak{U}))$
 é um ultrafiltro em 
 $Z$.
 A regra
 $(g \circ f)_* = g_* \circ f_*$
 é característica de mapas ``\textit{push forward}''.
 No caso de \textit{pullbacks}, a regra é
 $(g \circ f)^* = f^* \circ g^*$.
 
 \begin{theorem}
 \label{teo:3.2}
 Sejam 
 $(X, \tau_X), (Y, \tau_Y)$
 espaços topológicos e considere
 $f:X \to Y$
 qualquer. Afirmamos que
 $f$ é contínua se, e só se,
 $f_{*}(\mathfrak{U}) \to f(x)$
 em $Y$ para todo ultrafiltro
 $\mathfrak{U} \to x$ 
 em $X$.
 \end{theorem}
 
 \begin{proof}
 Suponha que 
 $f$
 é contínua e seja
 $\mathfrak{U}$
 um ultrafiltro em 
 $X$
 tal que
 $\mathfrak{U} \to x$.
 Para toda vizinhança aberta
 $V \subseteq Y$
 de
 $f(x)$,
 o conjunto
 $f^{-1}(V) \subseteq X$
 é uma vizinhança aberta de
 $x$.
 Temos
 $f^{-1}(V) \in \mathfrak{U}$
 e, portanto,
 $V \in f_*(\mathfrak{U})$,
 de modo que
 $f_*(\mathfrak{U}) \to f(x)$
 por definição.
 
 Reciprocamente, suponha que
 $f_*(\mathfrak{U}) \to f(x)$
 em 
 $Y$
 sempre que
 $\mathfrak{U}$
 é um ultrafiltro em
 $X$
 convergindo para
 $x$.
 Seja
 $V \subseteq Y$
 aberto.
 Para mostrar que a pré-imagem
 $f^{-1}(V) \subseteq X$
 é aberta, usaremos o Teorema \ref{teo:2.2}.
 Seja
 $\mathfrak{U}$
 um ultrafiltro qualquer em
 $X$
 convergindo para algum ponto
 $x \in f^{-1}(V)$.
 Por hipótese, 
 $f_*(\mathfrak{U}) \to f(x)$
 em
 $Y$.
 Como
 $V \subseteq Y$
 é uma vizinhança aberta de
 $f(x)$,
 temos
 $V \in f_*(\mathfrak{U})$,
 ou seja,
 $f^{-1}(V) \in \mathfrak{U}$.
 Pelo Teorema \ref{teo:2.2},
 $f^{-1}(V)$
 é aberto e, portanto,
 $f$
 é contínua.
 \end{proof}
 
 O Teorema \ref{teo:3.2} suscita uma comparação entre
 ultrafiltros e sequências.
 Recorde que para uma função contínua
 $f : X \to Y$,
 temos
 $f(x_n) \to f(x)$
 em
 $Y$
 sempre que
 $x_n \to x$
 em 
 $X$.
 A recíproca não vale sem que adicionemos algumas hipóteses 
 (por exemplo, que $X$ seja espaço métrico).
 Sequências da forma
 $x_1, x_2, x_3, \dots$
 são limitadas no sentido de que elas são, por definição, enumeráveis.
 Ultrafiltros, que não possuem tal restrição, são suficientes para 
 caracterizar topologias e continuidade em geral.
 
 %falar de filtros gerados
 \begin{theorem}[Filtros gerados]
 
 \todo{Eu pesquisei e coloquei esse teorema
 sobre filtros gerados, prof, mas não pensei
 ainda como ele poderia simplificar as coisas.}
 
 Sejam
 $X$
 um conjunto e
 $\mathcal{B} \subset \powerset X$.
 Temos que
 \begin{equation*}
     \mathfrak{F}
     =
     \setsuchthat{V \subseteq X}{\exists U \in \mathcal{B}, U \subseteq V}
 \end{equation*}
 é um filtro em 
 $X$
 se, e somente se,
 \begin{enumerate}
     \item $\forall V_1, V_2 \in \mathcal{B}, \exists U \in \mathcal{B} : U \subseteq V_1 \cap V_2$;
     \item $\varnothing \notin \mathcal{B}$;
     \item $\mathcal{B} \neq \varnothing$.
 \end{enumerate}
 Nesse caso, 
 $\mathcal{B}$
 é uma \textbf{base para o filtro}
 $\mathfrak{F}$,
 que é dito \textbf{gerado} por
 $\mathcal{B}$.
 \end{theorem}
 
 % demonstração de
 % https://proofwiki.org/wiki/Filter_Basis_Generates_Filter
 \begin{proof}
 Assumamos primeiro que
 $\mathfrak{F}$
 é um filtro em
 $X$.
 Temos então
 $X \in \mathfrak{F}$
 e, portanto,
 $\mathcal{B} \neq \varnothing$.
 Ademais, como
 $\varnothing \notin \mathfrak{F}$,
 temos que
 $\varnothing \notin \mathcal{B}$
 já que
 $\mathcal{B} \subseteq \mathfrak{F}$.
 Agora, sejam
 $V_1, V_2 \in \mathcal{B}$.
 Então 
 $V_1, V_2 \in \mathfrak{F}$
 e, portanto,
 $V = V_1 \cap V_2 \in \mathfrak{F}$
 já que 
 $\mathfrak{F}$
 é um filtro.
 A definição de
 $\mathfrak{F}$
 implica, então, que existe
 $U \subseteq \mathcal{B}$
 tal que 
 $U \subseteq V = V_1 \cap V_2$.
 
 Agora, assumamos que 
 $\mathcal{B}$
 satisfaz as três condições do enunciado.
 Para mostrar que 
 $\mathfrak{F}$
 é um filtro, note que como
 $\mathcal{B} \neq \varnothing$
 então existe 
 $B \in \mathcal{B}$.
 Como 
 $\mathcal{B} \subseteq \powerset X$,
 então
 $B \subseteq X$
 e daí, por definição de filtro,
 $X \in \mathfrak{F}$.
 Ademais, como
 $\varnothing \notin \mathcal{B}$,
 segue que
 $\varnothing \notin \mathfrak{F}$.
 Por fim, sejam
 $V_1, V_2 \in \mathfrak{F}$.
 Existem 
 $U_1, U_2 \in \mathcal{B}$
 tais que
 $U_1 \subseteq V_1$
 e
 $U_2 \subseteq V_2$.
 Além disso, existe
 $U \in \mathcal{B}$
 tal que
 $U \subseteq U_1 \cap U_2$,
 donde segue que
 $U \subseteq V_1 \cap V_2$
 e, portanto,
 $V_1 \cap V_2 \in \mathfrak{F}$.
 \end{proof}}
 
 {\section{Espaços produto com ultrafiltros}
\label{sec:esp_produto}
% Seção 4 ref
 %talvez tirar essa seção
 
 \todo{Eu lembro que o senhor disse que não
 precisava dessa seção, mas eu acho que seria
 legal se ela ficasse.
 Me ajudou a entender mais intuitivamente
 a topologia produto, e tem o Teorema 3.3.2
 que eu usei na demonstração do Tychonoff
 (tem um comentário sobre isso lá na demonstração).}
 
 Seja 
 $X_{\alpha}$
 uma família de espaços topológicos indexados por
 $\alpha \in A$.
 O \textbf{espaço produto}
 $X = \displaystyle{ \prod_{\alpha \in A} X_{\alpha} }$
 tem pontos da forma
 $x = (x_{\alpha})_{\alpha}$,
 sendo 
 $x_{\alpha} \in X_{\alpha}, \forall \alpha \in A.$
 Descreveremos a seguir a topologia natural neste conjunto,
 imposta pelas topologias individuais nos espaços
 $X_{\alpha}$.
 Antes, contudo, vamos deixar claro o que queremos dizer por um ponto
 $(x_{\alpha})_{\alpha}$.
 
 No caso de
 $|A| = n < \infty$,
 podemos tomar, sem perda de generalidade,
 $A = \{ 0, 1, 2, \dots, n-1 \}$.
 Nesse caso,
 \begin{equation*}
     X 
     = 
     X_0 \times X_1 \times \cdots \times X_{n-1}
     =
     \setsuchthat{ (x_0, x_1, \dots, x_{n-1}) }{ x_i \in X_i \forall i \in A }.
 \end{equation*}
 No caso particular em que
 $X_0 = X_1 = \cdots = X_{n-1}$,
 obtemos
 \begin{equation*}
     X 
     = 
     \underbrace{ X_0 \times X_0 \times \cdots \times X_0 }_{n \text{ vezes}}
     =
     X_0^n.
 \end{equation*}
 Todo ponto
 $x = (x_0, x_1, \dots, x_{n-1}) \in X$
 pode então ser visto como uma função
 $f: A \to X_0$.
 
 Prosseguindo para produtos enumeráveis,
 vamos usar
 $\omega = \{ 0, 1, 2, \dots \}$
 como nosso conjunto de índices.
 Agora, temos um produto enumerável de conjuntos dado por
 \begin{equation*}
     X
     =
     \prod_{i \in \omega} X_i
     =
     X_0 \times X_1 \times X_2 \times \cdots
     =
     \setsuchthat{ (x_0, x_1, x_2, \dots) }{ x_i \in X_i \forall i \in \omega }.
 \end{equation*}
 Aqui, cada ponto
 $x \in X$
 é uma sequência infinita,
 identificada com a correspondência
 $i \mapsto x_i$.
 Como casos especiais em que todos os espaços da família são iguais, 
 temos
 \begin{equation*}
     \reals^{\omega}
     =
     \reals \times \reals \times \reals \times \cdots,
 \end{equation*}
 o conjunto de todas as sequências infinitas enumeráveis de números reais,
 e
 \begin{equation*}
     2^{\omega}
     =
     \{ 0,1 \}^{\omega}
     =
     \{ 0,1 \} \times \{ 0,1 \} \times \{ 0,1 \} \times \cdots,
 \end{equation*}
 o conjunto de todas as sequências binárias infinitas enumeráveis.
 
 Para um conjunto $A$ de cardinalidade arbitrária,
 o produto geral
 $X = \displaystyle{ \prod_{\alpha \in A} X_{\alpha} }$
 pode ser pensado como o conjunto de todas as funções definidas em
 $A$
 que levam
 $\alpha \mapsto x_{\alpha} \in X_{\alpha}$
 para cada
 $\alpha \in A.$
 Podemos denotar os pontos por 
 $x = (x_{\alpha})_{\alpha \in A}$.
 Novamente considerando o caso particular em que todos os
 $X_{\alpha}$'s
 são iguais a um mesmo espaço
 $X_0$,
 temos
 \begin{equation*}
     X
     =
     X_0^A
     =
     \{ \text{funções } A \to X_0 \}.
 \end{equation*}
 Por exemplo, o conjunto
 $\reals^{\reals}$
 é simplesmente o conjunto de todas as funções
 $\reals \to \reals$.
 
 Todo espaço produto
 $X = \displaystyle{ \prod_{\alpha \in A} X_{\alpha} }$
 vem naturalmente equipado com projeções nos vários fatores;
 elas são as sobrejeções dadas por
 \begin{equation*}
     \functionarray{\pi_{\alpha}}{X}{X_{\alpha}}{x}{x_{\alpha}}.
 \end{equation*}
 Pensando em um ponto
 $x \in X$
 como uma função, a $\alpha$-ésima coordenada
 $\pi_{\alpha}(x) = x_{\alpha}$
 é simplesmente o valor daquela função no \textit{input}
 $\alpha \in A$.
 
 A partir daí, a topologia natural que mencionamos para o espaço produto
 é a chamada \textbf{topologia produto}, que é a topologia mais fraca
 na qual cada uma das projeções
 $\pi_{\alpha} : X \to X_{\alpha}$
 é contínua. 
 Equivalentemente, uma base para a topologia produto em 
 $X$
 é a coleção de abertos da forma
 \begin{equation*}
     \prod_{\alpha \in A} U_{\alpha}
 \end{equation*}
 onde cada
 $U_{\alpha} \subseteq X_{\alpha}$
 é aberto e
 $U_{\alpha} \neq X_{\alpha}$
 para um número finito de índices.
 
 Por outro lado, a \textbf{topologia caixa}
 \footnote{Tradução literal do termo em inglês \textit{box topology}.}
 no espaço produto 
 $X$
 é a topologia cuja base consiste de todos os conjuntos da forma
 \begin{equation*}
     \prod_{\alpha \in A} U_{\alpha}
 \end{equation*}
 onde
 $U_{\alpha} \subseteq X_{\alpha}$
 apenas.
 Se $|A| < \infty$, então as duas topologias coincidem.
 Entretanto, em geral a topologia caixa é mais forte que a topologia produto
 (em geral forte demais para ser útil).
 A menos que seja especificado, 
 tomaremos a topologia produto para o espaço produto $X$.
 
 Como um exemplo, considere a sequência de pontos
 $v_1, v_2, v_3, \dots \in [0,1]^{\omega}$
 dada por
 \begin{align*}
     v_1 &= (0, 1, 1, 1, 1, 1, \dots), \\
     v_2 &= (0, 0, 1, 1, 1, 1, \dots), \\
     v_3 &= (0, 0, 0, 1, 1, 1, \dots), \\
     v_4 &= (0, 0, 0, 0, 1, 1, \dots),
 \end{align*}
 e assim por diante. É natural supor que
 $v_n$
 converge para o ponto
 $\mathbf{0} = (0,0,0,0,0,0,\dots)$
 em
 $[0,1]^{\omega}$,
 e isso de fato é verdade na topologia produto:
 uma vizinhança aberta de 
 $\mathbf{0}$ 
 nessa topologia tem a forma
 \begin{equation*}
     U
     =
     U_0 \times U_1 \times U_2 \times \cdots \times U_{m-1} 
     \times [0,1] \times [0,1] \times [0,1] \times \cdots,
 \end{equation*}
 em que cada
 $U_i$
 é uma vizinhança aberta de
 $0$.
 Como
 $v_n \in U$
 sempre que 
 $n \geq m$,
 temos
 $v_n \to \mathbf{0}$.
 Na topologia caixa, por outro lado, a sequência
 $v_n$
 não converge: considere por exemplo a vizinhança
 \begin{equation*}
     U'
     =
     [0, 1/2] \times [0, 1/2] \times [0, 1/2] \times [0, 1/2] \times \cdots,
 \end{equation*}
 uma vizinhança aberta de
 $\mathbf{0}$
 na topologia caixa.
 Note que
 $v_n$
 nunca entra em
 $U'$,
 não importa quão grande seja
 $n$.
 Uma primeira justificativa para a não convergência seria que
 as coordenadas de
 $v_n$
 convergem ``pontualmente'', e não ``uniformemente'', para 
 $0$.
 Contudo, o problema na verdade é pior:
 considere a sequência
 \begin{align*}
     w_1 &= (0, 1, 1, 1, 1, 1, \dots), \\
     w_2 &= (0, 0, 1/2, 1/2, 1/2, 1/2, \dots), \\
     w_3 &= (0, 0, 0, 1/4, 1/4, 1/4, \dots), \\
     w_4 &= (0, 0, 0, 0, 1/8, 1/8, \dots),
 \end{align*}
 e assim por diante,
 que converge para
 $\mathbf{0}$
 na topologia produto.
 As coordenadas convergem ``uniformemente'' para 0,
 e o fazem razoavelmente rápido,
 mas
 $w_n$
 nunca entra em 
 \begin{equation*}
     [0,1] \times [0,1/2] \times [0,1/3] \times [0,1/5] \times [0,1/9] \times \cdots,
 \end{equation*}
 logo
 $w_n \not\to \mathbf{0}$
 na topologia caixa.
 Essa topologia contém abertos demais, i.e.,
 está muito próxima da topologia discreta para nos ser de grande utilidade.
 
 Generalizando os exemplos acima,
 vemos que uma sequência de pontos de
 $X = \displaystyle{ \prod_{\alpha \in A} X_{\alpha} }$
 converge se, e somente se, convergir ``coordenada a coordenada'':
 
 \begin{theorem}
 \label{teo:4.1}
 Seja 
 $x_n = (x_{n, \alpha})_{\alpha}$
 uma sequência de pontos em 
 $X = \displaystyle{ \prod_{\alpha} X_{\alpha}}$.
 Seja também 
 $a = (a_{\alpha})_{\alpha} \in X$.
 Então
 $x_n \to a$
 em $X$ se, e somente se,
 $x_{n, \alpha} \to a_{\alpha}$
 para todo $\alpha$ quando $n \to \infty$.
 \end{theorem}
 
 \begin{proof}
 Suponha primeiramente que 
 $x_n \to a$.
 Como cada projeção
 $\pi_{\alpha}$
 é contínua, isso implica que
 $x_{n, \alpha} = \pi(x_n) \to \pi_{\alpha}(a) = a_{\alpha}$
 para todo $\alpha$.
 
 Para mostrar a recíproca, basta considerarmos uma vizinhança aberta
 $U$
 de 
 $a \in X$.
 Esse conjunto tem a forma
 $\displaystyle{ \prod_{\alpha} U_{\alpha} }$,
 sendo cada
 $U_{\alpha}$
 uma vizinhança aberta de
 $a_{\alpha} \in X_{\alpha}$
 e
 $U_{\alpha} = X_{\alpha}$
 para todo 
 $\alpha \notin \{ \alpha_1, \alpha_2, \dots, \alpha_m \}$.
 Sob a hipótese de que
 $x_{\alpha, n} \to a_{\alpha}$
 para todo
 $\alpha \in A$,
 existem constantes
 $N_1, N_2, \dots, N_m$
 tais que para todo
 $i \in \{ 1, 2, \dots, m \}$
 temos
 $x_{n, \alpha_i} \in U_{\alpha_i}$
 sempre que
 $n > N_i$.
 Seja
 $N = \max\{ N_1, \dots, N_m \}$.
 Para todo
 $n > N$
 e para todo
 $\alpha \in A$,
 temos
 $x_{n, \alpha} \in U_{\alpha}$:
 se
 $\alpha \in \{ \alpha_1, \dots, \alpha_m \}$,
 esse fato segue da escolha de 
 $n > N \geq N_i$;
 do contrário, esse fato segue simplesmente porque
 $x_{\alpha} \in U_{\alpha} = X_{\alpha}$.
 \end{proof}
 
 A topologia produto também pode ser caracterizada 
 utilizando ultrafiltros, como mostra o teorema a seguir.
 
 \begin{theorem}
 \label{teo:4.2}
 Sejam
 $\mathfrak{U}$
 um ultrafiltro em
 $X = \displaystyle{ \prod_{\alpha} X_{\alpha}}$
 e
 $x = (x_{\alpha})_{\alpha} \in X$.
 Então
 $\mathfrak{U} \to x$
 se, e somente se
 $(\pi_{\alpha})_{*}(\mathfrak{U}) \to x_{\alpha}$
 em cada $X_{\alpha}$.
 \end{theorem}
 
 \begin{proof}
 Se
 $\mathfrak{U} \to x$,
 então pelo Teorema \ref{teo:3.2} para cada
 $\alpha \in A$
 temos
 $(\pi_{\alpha})_*(\mathfrak{U}) \to x_{\alpha}$.
 
 Reciprocamente, suponha que para cada
 $\alpha \in A$
 temos
 $(\pi_{\alpha})_*(\mathfrak{U}) \to x_{\alpha}$.
 Seja
 $u \subseteq X$
 uma vizinhança aberta de
 $x$.
 Devemos mostrar que 
 $U \in \mathfrak{U}.$
 Para isso, basta considerar abertos da sub base da forma
 $U = \pi_{\alpha}^{-1}(U_{\alpha})$
 onde cada
 $U_{\alpha} \subseteq X_{\alpha}$
 é uma vizinhança aberta de
 $x_{\alpha}$.
 Nesse caso,
 $U_{\alpha} \in (\pi_{\alpha})_*(\mathfrak{U})$,
 de modo que
 $U = \pi_{\alpha}^{-1}(U_{\alpha}) \in \mathfrak{U}$
 como desejado.
 Como $\mathfrak{U}$ é fechado por interseções finitas 
 e para a operação de contém, esse resultado vale para
 uma vizinhança aberta arbitrária de
 $x$,
 nos dando
 $\mathfrak{U} \to x$.
 \end{proof}}}

{\chapter{Compacidade}

{\section{Compacidade com filtros}
\label{sec:compac_filtros}

 %mostrar equivalência entre as diferentes definições
 
 A definição ``standard'' é dada 
 em termos de coberturas, e foi usada
 na demonstração do Teorema \ref{teo:2.1}.
 De fato, o item (b) desse teorema nos mostra que
 uma definição alternativa de compacidade em termos
 de filtros poderia ser
 
 \begin{definition}[Compacidade com filtros]
 \label{def:compac_filtros}
 Todo ultrafiltro converge para pelo menos
 um ponto.
 \end{definition}
 
 %Se F é um ultrafiltro, então existe x tal que V(x) \subset F.
 
%  \begin{definition}[Compacidade com abertos]
%  \label{def:compac_abertos}
%  Dizemos que o espaço topológico
%  $X$
%  é compacto (com respeito a uma determinada topologia)
%  se toda cobertura aberta
%  $\mathcal{A}$
%  de 
%  $X$
%  possui subcobertura finita.
%  \end{definition}}

{\section{O teorema de Tychonoff com filtros}
\label{sec:tychonoff}

 \begin{theorem}[Thychonoff]
 \label{teo:tychonoff}
 Dada uma família
 $X_{\alpha}$
 de espaços topológicos compactos,
 o espaço produto
 $X = \displaystyle{ \prod_{\alpha} X_{\alpha} }$
 também é compacto com respeito à topologia produto.
 \end{theorem}
 
 Apresentamos aqui uma demonstração utilizando filtros.
 Em seguida, apresentamos e demonstramos o Teorema
 da sub-base de Alexander e utilizamos esse resultado
 para dar um prova alternativa do Teorema de Tychonoff.
 
 \begin{proof}
 \todo{Aqui eu coloquei uma demonstração
 que aproveita a caracterização de compacidade
 com filtros que eu tinha colocado antes.
 Não sei se era exatamente isso que você
 tinha em mente, prof, mas mesmo que não
 seja achei o argumento muito elegante :-)}
 % demonstrar usando filtros
 % a demonstração tem dois passos:
 
 % 1. any image of an ultrafilter is an ultrafilter 
 % (in particular, any projection from a product space)
 
 % -> esse ponto é a proposição do ultrafiltro push forward, então done
 
 % 2. any filter in the product space converges 
 % if and only if all its projections converge
 % -> esse é o teorema 3.3.2 do Cap 3
 
 Da Proposição \ref{prop:imag_ultrafiltro},
 sabemos que imagem de ultrafiltro é ultrafiltro,
 então em particular as imagens de ultrafiltros no
 espaço produto pelas projeções
 $\pi_{\alpha}$
 em cada
 $X_{\alpha}$
 são também ultrafiltros.
 
%  O Teorema \ref{teo:4.2} 
%  nos diz que os ultrafiltros
%  $(\pi_{\alpha})_*(\mathfrak{U})$
%  convergem para pelo menos um ponto em cada
%  $X_{\alpha}$
%  se, e somente se, o ultrafiltro
%  $\mathfrak{U}$
%  no espaço produto converge para pelo menos um ponto.
 
 Ora, então o Teorema \ref{teo:2.1}
 nos permite concluir que 
 $X$
 é compacto já que, por hipótese, os
 $X_{\alpha}$ 
 são compactos e, portanto, os
 $(\pi_{\alpha})_*(\mathfrak{U})$
 convergem para pelo menos um ponto pelo
 Teorema \ref{teo:4.2}.
 \end{proof}}

{\section{O teorema da sub-base de Alexander}
\label{sec:alexander}

 \todo{Nessa seção eu ainda estou com as demonstrações
 inalteradas, tem de justificar melhor o uso do
 Lema de Zorn na demonstração do Teorema da Sub-Base.
 Além disso, também vou pensar em como fazer
 com filtros e sem o Lema de Zorn.}
 
 \begin{theorem}[Alexander]
 \label{teo:alexander}
 Seja
 $(X, \tau)$
 um espaço topológico.
 Se 
 $X$
 tem uma sub base
 $\mathcal{C}$
 tal que toda cobertura de
 $X$
 por elementos de
 $\mathcal{C}$
 tem uma subcobertura finita, então
 $X$
 é compacto.
 \end{theorem}
 
 % pensar em como fazer com filtros e sem
 % lema de Zorn
 
 % enxugar a demonstração com a do Matheus
 % lá do fórum
 
 % prova original enxugada
 \begin{proof}
 Suponha que 
 $X$
 não é compacto.
 Seja
 $\mathscr{F}$
 a coleção de todas as coberturas abertas de
 $X$
 que não têm subcobertura finita.
 Note que
 $\mathscr{F} \neq \varnothing$
 pois
 $X$
 não é compacto,
 e que essa família é parcialmente ordenada
 pela relação de inclusão.
 Daí, o Lema de Zorn garante que existe
 $\mathcal{M} \subseteq \mathscr{F}$
 maximal (com respeito à relação de inclusão).
 
 De fato, seja
 \begin{equation*}
     \mathscr{F}'
     =
     \setsuchthat{
     \mathcal{F}_{\lambda}
     }
     {
     \lambda \in \Lambda
     } \subseteq \mathscr{F}
 \end{equation*}
 um conjunto totalmente ordenado. Defina
 \begin{equation*}
     \mathcal{F} 
     = 
     \bigcup_{\Lambda} \mathcal{F}_{\lambda}.
 \end{equation*}
 Temos que 
 $\mathcal{F}$
 é uma cota superior de 
 $\mathscr{F}'$
 pois contém todo
 $\mathcal{F}_{\lambda}$
 e também que
 $\mathcal{F}$
 é uma cobertura aberta de
 $X$.
 
 Se tivéssemos
 $\mathcal{F} \notin \mathscr{F}'$,
 então existiriam 
 $F_1, \dots, F_n \in \mathcal{F}$
 cobrindo 
 $X$.
 Ora, mas cada um dos
 $F_j$ 
 pertence a algum dos
 $\mathcal{F}_{\lambda}$.
 Digamos que
 $F_j \in \mathcal{F}_{\lambda_j}$.
 Como
 $\mathscr{F}'$
 é totalmente ordenado,
 então
 $\{ \mathcal{F}_{\lambda_1}, \dots, \mathcal{F}_{\lambda_n} \}$
 também é.
 Sem perda de generalidade, suponhamos que
 $\mathcal{F}_{\lambda_n}$
 contém todos os
 $F_j$.
 Ora, então
 $\mathcal{F}_{\lambda_n}$
 é um elemento de
 $\mathscr{F}'$
 que contém uma subcobertura finita, ou seja,
 $\mathscr{F}'$
 tem uma subcobertura finita de 
 $X$, 
 absurdo.
 Portanto, 
 $\mathcal{F} \in \mathscr{F}'$.
 
 Seja 
 $\mathcal{S} = \mathcal{M} \cap \mathcal{C}$
 o conjunto dos abertos de
 $\mathcal{M}$
 que estão na sub-base.
 Vamos mostrar que
 $\mathcal{S}$
 cobre
 $X$,
 o que implicará que
 $\mathcal{M}$
 deve conter uma subcobertura finita de
 $X$,
 já que
 $\mathcal{S} \subset \mathcal{C}$
 e
 $\mathcal{S} \subseteq \mathcal{M}$,
 e, daí, teremos a conclusão absurda de que
 $\mathcal{M} \notin \mathscr{F}$,
 concluindo então que 
 $X$
 é, de fato, compacto.
 
 Suponhamos então que
 $\mathcal{S}$
 não cobre 
 $X$.
 Existe, então,
 $x \in X \setminus \bigcup \mathcal{S}$.
 Como
 $\mathcal{M}$
 cobre
 $X$, 
 existe
 $U \in \mathcal{M}$
 tal que
 $x\in U$
 e, como
 $\mathcal{C}$
 é sub-base, existem
 $C_1, \dots, C_n \in \mathcal{C}$
 tais que
 $C = \displaystyle{\bigcap_{i}^n C_i } \subseteq U$.
 Note que nenhum dos
 $C_i$
 está em
 $\mathcal{M}$,
 pois se um deles estivesse, teríamos
 $x\in C \in \mathcal{M}$,
 contrariando o fato de que 
 $x$
 não é coberto por 
 $\mathcal{S}$.
 
 Como 
 $\mathcal{M}$
 é maximal, segue que cada uma das coleções
 $\mathcal{M} \cup \{ C_i \}$
 possui subcobertura finita de
 $X$.
 Denote-a por
 $\mathcal{M}_i \cup \{ C_i \}$,
 sendo
 $\mathcal{M}_i \subseteq \mathcal{M}$.
 Temos, então, que
 \begin{equation*}
     \bigcup_i \mathcal{M}_i \cup \{ C \}
 \end{equation*}
 é uma cobertura aberta finita de 
 $X$,
 de modo que
 \begin{equation*}
     \bigcup_i \mathcal{M}_i \cup \{ U \}
 \end{equation*}
 também o é, já que
 $C \subseteq U$.
 Mas
 $U \in \mathcal{M}$,
 absurdo.
 
 Portanto,
 $\mathcal{S}$
 deve cobrir 
 $X$
 e, consequentemente,
 $X$
 é compacto.
 \end{proof}
 
 % prova original
%  \begin{proof}
%  O argumento é por absurdo.
%  Suponhamos que
%  $X$
%  não é compacto
%  (de modo que $X$ é infinito),
%  mas que toda cobertura sub básica por elementos de
%  $\mathcal{S}$
%  tem uma subcobertura finita.
%  Denote por
%  $\mathbb{S}$
%  o conjunto de todas as coberturas abertas de 
%  $X$
%  que não têm subcobertura finita de
%  $X$.
%  Ordene 
%  $\mathbb{S}$
%  parcialmente usando a relação de inclusão de conjuntos e
%  aplique o Lema de Zorn para encontrar um elemento
%  $\mathscr{C} \in \mathbb{S}$
%  maximal. 
%  Note que
%  \begin{enumerate}
%      \item como
%      $\mathscr{C} \in \mathbb{S}$,
%      então por definição de
%      $\mathbb{S}$
%      temos que
%      $\mathscr{C}$
%      é uma cobertura aberta de
%      $X$
%      que não tem subcobertura finita de 
%      $X$
%      (isso implica, em particular, que $\mathscr{C}$ é infinito)
     
%      \item a maximalidade de
%      $\mathscr{C}$
%      em
%      $\mathbb{S}$
%      implica que se
%      $V$
%      é um aberto de
%      $X$
%      tal que
%      $V \notin \mathscr{C}$,
%      então
%      $\mathscr{C} \cup \{ V \}$
%      tem subcobertura finita, que deve necessariamente ser da forma
%      $\{ V \} \cup \mathscr{C}_V$
%      para algum subconjunto finito
%      $\mathscr{C}_V$
%      de
%      $\mathscr{C}$.
%      Como sugere a notação, esse subconjunto finito depende de
%      $V$.
%  \end{enumerate}
%  Começaremos mostrando que
%  $\mathscr{C} \cap \mathcal{S}$
%  \textbf{não} cobre
%  $X$.
%  Suponhamos o contrário, o que implica, em particular, que
%  $\mathscr{C} \cap \mathcal{S}$
%  é uma cobertura de 
%  $X$
%  por elementos de
%  $\mathcal{S}$.
%  A hipótese do teorema em
%  $\mathcal{S}$
%  implica que existem um subconjunto finito de
%  $\mathscr{C} \cap \mathcal{S}$
%  que cobre
%  $X$,
%  que também é, simultaneamente, uma subcobertura finita de
%  $X$
%  por elementos de
%  $\mathscr{C}$,
%  já que
%  $\mathscr{C} \cap \mathcal{S} \subseteq \mathscr{C}$.
%  Mas isso contradiz o fato de que
%  $\mathscr{C} \in \mathcal{S}$,
%  provando que
%  $\mathscr{C} \cap \mathcal{S}$
%  não cobre 
%  $X$.
 
%  Como 
%  $\mathscr{C} \cap \mathcal{S}$
%  não cobre
%  $X$,
%  existe
%  $x \in X$
%  que não é coberto por
%  $\mathscr{C} \cap \mathcal{S}$,
%  ou seja, tal que
%  $x$
%  não pertence a nenhum elemento de
%  $\mathscr{C} \cap \mathcal{S}$.
%  Mas como
%  $\mathscr{C}$
%  cobre
%  $X$,
%  também existe
%  $U \in \mathscr{C}$
%  tal que
%  $x \in U$.
%  Como
%  $\mathcal{S}$
%  é uma sub base geradora da topologia de
%  $X$,
%  segue da definição de topologia gerada que
%  existe uma coleção finita de abertos da sub base
%  $S_1, \dots, S_n \in \mathcal{S}$
%  tal que
%  \begin{equation*}
%      x \in S_1 \cap \cdots \cap S_n \subseteq U.
%  \end{equation*}
%  Agora, vamos mostrar por absurdo que
%  $S_i \notin \mathscr{C}$
%  para todo
%  $i = 1, 2, \dots, n$.
%  Se 
%  $i$
%  fosse tal que 
%  $S_i \in \mathscr{C}$,
%  então teríamos
%  $S_i \in \mathscr{C} \cap \mathcal{S}$,
%  de modo que
%  $x \in S_i$
%  implicaria que
%  $x$
%  é coberto por
%  $\mathscr{C} \cap \mathcal{S}$,
%  contradizendo a maneira como 
%  $x$
%  foi escolhido.
 
%  Como mencionado anteriormente, a maximalidade de
%  $\mathscr{C}$
%  em
%  $\mathcal{S}$
%  implica que para todo
%  $i = 1, 2, \dots, n$, 
%  existe um subconjunto finito
%  $\mathscr{C}_{S_i}$
%  de
%  $\mathscr{C}$
%  tal que
%  $\{ S_i \} \cup \mathscr{C}_{S_i}$
%  é uma cobertura de 
%  $X$.
%  Defina
%  \begin{equation*}
%      \mathscr{C}_F 
%      = 
%      \mathscr{C}_{S_1} \cup \cdots \cup \mathscr{C}_{S_n},
%  \end{equation*}
%  que é um subconjunto finito de
%  $\mathscr{C}$
%  pois cada um dos
%  $\mathscr{C}_{S_i}$
%  o são. 
%  Note que para cada
%  $i = 1, 2, \dots, n$,
%  a união
%  $\{ S_i \} \cup \mathscr{C}_F$
%  é uma cobertura finita de 
%  $X$. 
%  Portanto, trocaremos os
%  $\mathscr{C}_{S_i}$
%  por
%  $\mathscr{C}_F$.
 
%  Denote por
%  $\bigcup \mathscr{C}_F$
%  a união de todos os conjuntos em
%  $\mathscr{C}_F$
%  (que é um aberto de $X$)
%  e denote
%  $Z = X \setminus \bigcup \mathscr{C}_F$.
%  Observe que para qualquer
%  $A \subseteq X$, 
%  temos que
%  $\{ A \} \cup \bigcup \mathscr{C}_F$
%  cobre
%  $X$
%  se, e somente se,
%  $Z \subseteq A$.
%  Em particular, para todo
%  $i = 1, 2, \dots, n$,
%  o fato de que
%  $\{ S_i \} \cup \bigcup \mathscr{C}_F$
%  cobre
%  $X$
%  implica que
%  $Z \subseteq S_i$.
%  Portanto,
%  $Z \subseteq S_1 \cap \cdots \cap S_n$.
%  Lembrando que
%  $S_1 \cap \cdots \cap S_n \subseteq U$,
%  temos
%  $Z \subseteq U$,
%  o que é equivalente a
%  $\{ U \} \cup \bigcup \mathscr{C}_F$
%  ser uma cobertura de
%  $X$.
%  Ademais, 
%  $\{ U \} \cup \bigcup \mathscr{C}_F$
%  é uma cobertura finita de 
%  $X$
%  tal que
%  $\{ U \} \cup \bigcup \mathscr{C}_F \subseteq \mathscr{C}$.
%  Portanto, 
%  $\mathscr{C}$
%  tem uma subcobertura finita de
%  $X$,
%  contradizendo o fato de que
%  $\mathscr{C} \in \mathbb{S}.$
%  Logo, a suposição inicial de que
%  $X$
%  não era compacto está incorreta, mostrando que
%  $X$
%  é compacto.
%  \end{proof}
 
 Note que a recíproca do teorema também vale:
 se
 $X$
 é compacto e 
 $\mathcal{S}$
 é uma sub base de
 $X$,
 então toda cobertura de
 $X$
 por elementos de
 $\mathcal{S}$
 tem subcobertura finita.
 Ela pode ser demonstrada tomando
 $\mathcal{S} = \tau$
 (já que toda topologia é sub base de si mesma).
 
 Com esse teorema,
 podemos demonstrar o Teorema \ref{teo:alexander}
 de maneira alternativa, sem filtros, como segue.
 
 \begin{proof}[Prova alternativa do Teorema \ref{teo:tychonoff}]
 A topologia produto em
 $X$
 tem, por definição, uma sub base que consiste de
 conjuntos ``cilindros'', que são as imagens inversas
 pelas projeções de abertos em cada fator.
 Dada uma família
 $C$
 de elementos da sub base de
 $X$
 que não tem subcobertura finita,
 podemos particionar
 $C = \displaystyle{ \bigcup_{\alpha} C_{\alpha} }$
 em subfamílias que consistem exatamente dos conjuntos
 cilindros correspondentes a um dado fator de
 $X$.
 Por hipótese, se
 $C_{\alpha} \neq \varnothing$,
 então
 $C_{\alpha}$
 não tem subcobertura finita.
 Por serem conjuntos cilindros, isso significa que
 suas projeções em
 $X_{\alpha}$
 não tem subcobertura finita, e como cada
 $X_{\alpha}$
 é compacto, podemos encontrar
 $x_{\alpha} \in X_{\alpha}$
 que não é coberto pelas projeções de
 $C_{\alpha}$
 em
 $X_{\alpha}$.
 Mas então
 $(x_{\alpha})_{\alpha} \in \displaystyle{ \prod_{\alpha} X_{\alpha} }$
 não é coberto por
 $C$, 
 absurdo.
 \end{proof}}}

%{\chapter{Redes}}}

  \bibliographystyle{amsplain}
  \bibliography{trabalho_final_de_topologia_geral}
\end{document}
